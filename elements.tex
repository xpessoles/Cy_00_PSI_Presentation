% Options for packages loaded elsewhere
\PassOptionsToPackage{unicode}{hyperref}
\PassOptionsToPackage{hyphens}{url}
%
\documentclass[
]{article}
\usepackage{amsmath,amssymb}
\usepackage{lmodern}
\usepackage{iftex}
\ifPDFTeX
  \usepackage[T1]{fontenc}
  \usepackage[utf8]{inputenc}
  \usepackage{textcomp} % provide euro and other symbols
\else % if luatex or xetex
  \usepackage{unicode-math}
  \defaultfontfeatures{Scale=MatchLowercase}
  \defaultfontfeatures[\rmfamily]{Ligatures=TeX,Scale=1}
\fi
% Use upquote if available, for straight quotes in verbatim environments
\IfFileExists{upquote.sty}{\usepackage{upquote}}{}
\IfFileExists{microtype.sty}{% use microtype if available
  \usepackage[]{microtype}
  \UseMicrotypeSet[protrusion]{basicmath} % disable protrusion for tt fonts
}{}
\makeatletter
\@ifundefined{KOMAClassName}{% if non-KOMA class
  \IfFileExists{parskip.sty}{%
    \usepackage{parskip}
  }{% else
    \setlength{\parindent}{0pt}
    \setlength{\parskip}{6pt plus 2pt minus 1pt}}
}{% if KOMA class
  \KOMAoptions{parskip=half}}
\makeatother
\usepackage{xcolor}
\IfFileExists{xurl.sty}{\usepackage{xurl}}{} % add URL line breaks if available
\IfFileExists{bookmark.sty}{\usepackage{bookmark}}{\usepackage{hyperref}}
\hypersetup{
  hidelinks,
  pdfcreator={LaTeX via pandoc}}
\urlstyle{same} % disable monospaced font for URLs
\usepackage{longtable,booktabs,array}
\usepackage{calc} % for calculating minipage widths
% Correct order of tables after \paragraph or \subparagraph
\usepackage{etoolbox}
\makeatletter
\patchcmd\longtable{\par}{\if@noskipsec\mbox{}\fi\par}{}{}
\makeatother
% Allow footnotes in longtable head/foot
\IfFileExists{footnotehyper.sty}{\usepackage{footnotehyper}}{\usepackage{footnote}}
\makesavenoteenv{longtable}
\usepackage{graphicx}
\makeatletter
\def\maxwidth{\ifdim\Gin@nat@width>\linewidth\linewidth\else\Gin@nat@width\fi}
\def\maxheight{\ifdim\Gin@nat@height>\textheight\textheight\else\Gin@nat@height\fi}
\makeatother
% Scale images if necessary, so that they will not overflow the page
% margins by default, and it is still possible to overwrite the defaults
% using explicit options in \includegraphics[width, height, ...]{}
\setkeys{Gin}{width=\maxwidth,height=\maxheight,keepaspectratio}
% Set default figure placement to htbp
\makeatletter
\def\fps@figure{htbp}
\makeatother
\setlength{\emergencystretch}{3em} % prevent overfull lines
\providecommand{\tightlist}{%
  \setlength{\itemsep}{0pt}\setlength{\parskip}{0pt}}
\setcounter{secnumdepth}{-\maxdimen} % remove section numbering
\ifLuaTeX
  \usepackage{selnolig}  % disable illegal ligatures
\fi

\author{}
\date{}

\begin{document}

\includegraphics[width=9.01493in,height=4.10448in]{media/image1.png}

\hypertarget{introduction-guxe9nuxe9rale}{%
\section{Introduction générale}\label{introduction-guxe9nuxe9rale}}

L'objectif de ce document est de procurer une introduction aux
technologies couramment utilisées en Sciences Industrielles pour
l'Ingénieur. Il contient des descriptions de composants ainsi que des
éléments permettant de comprendre leur fonctionnement et leur
modélisation. Afin de structurer cet inventaire et de faciliter son
utilisation nous découperons le document selon les principes de la
chaine fonctionnelle. On rappelle que l'objet de la chaine fonctionnelle
est de décrire les composants et les flux (matière, énergie,
information) qui permettent à un système d'ajouter une valeur à une
matière d'œuvre. Les deux premiers chapitres couvrent les fonctions
\emph{acquérir} et \emph{traiter} de la chaine d'information.

Les fonctions \emph{stocker, alimenter, moduler, convertir} et
\emph{transmettre} sont l'objet des 5 chapitres suivant. La fonction
d'action n'est pas détaillée car elle est pratiquement unique à chaque
système en fonction de son cas d'utilisation.

\emph{Remarques~:} La fonction \emph{communiquer} fait l'objet d'un
document séparé. En PCSI-PSI les détails de la couche de communication
ne sont pas abordés. En général le bloc communication concerne un
procédé de et non un composant spécifique entre l'unité de traitement et
le préactionneur (modulation, commutation).

\emph{Exemple de protocole entre un microcontrôleur et un
préactionneur~:} rapport cyclique d'un signal PWM pour communiquer une
grandeur ou bus I2C pour échanger des informations plus complexes.

Lorsqu'un système fait appel à plusieurs unités de traitement (un
ordinateur déporté et un microcontrôleur par exemple) la qualité de la
connexion et l'architecture du système d'information permettent en
général de considérer que l'impact des procédés de communication est
négligeable sur la modélisation du comportement du système

La fonction «~restituer/Interface homme machine~» fait référence aux
voyants, afficheurs, écrans, haut-parleurs...

\includegraphics[width=6.83747in,height=1.96354in]{media/image5.png}

\hypertarget{table-des-matiuxe8res}{%
\section{Table des matières}\label{table-des-matiuxe8res}}

\protect\hyperlink{introduction-guxe9nuxe9rale}{1 Introduction générale
1}

\protect\hyperlink{fonction-acquuxe9rir}{2 Fonction acquérir 5}

\protect\hyperlink{nature-des-informations-grandeur-analogique-numuxe9rique-binaire}{2.1
Nature des informations \emph{Grandeur Analogique, Numérique, Binaire}
5}

\protect\hyperlink{signal-dans-un-systuxe8me-numuxe9rique-uxe9chantillonnage-et-ruxe9solution}{2.2
signal~dans un système numérique : échantillonnage et résolution 6}

\protect\hyperlink{caractuxe9ristiques-des-capteurs}{2.3
Caractéristiques des capteurs 7}

\protect\hyperlink{duxe9finitions}{2.3.1 Définitions \emph{Mesurande,
Fidélité, Justesse, Précision, Exactitude, Etalonnage, Calibration,
Resolution, Sensibilité, Rapidité,} 7}

\protect\hyperlink{probluxe8mes-de-mesure-offset-gain-linuxe9arituxe9}{2.3.2
Problèmes de mesure \emph{Offset, Gain, Linéarité} 8}

\protect\hyperlink{les-duxe9tecteurs-capteurs-logiques}{2.4 Les
détecteurs -- Capteurs logiques 9}

\protect\hyperlink{duxe9tecteurs-uxe0-contactfin-de-course}{2.4.1
Détecteurs à contact/Fin de course 9}

\protect\hyperlink{capteur-pneumatique}{2.4.2 Capteur pneumatique 9}

\protect\hyperlink{interrupteur-uxe0-lame-souples-ils}{2.4.3
Interrupteur à lame souples (ILS) 10}

\protect\hyperlink{duxe9tecteur-photouxe9lectrique-barrage-reflexproximituxe9}{2.4.4
Détecteur photoélectrique barrage, reflex/proximité 10}

\protect\hyperlink{duxe9tecteur-inductif}{2.4.5 Détecteur inductif 11}

\protect\hyperlink{duxe9tecteur-capacitif}{2.4.6 Détecteur capacitif 11}

\protect\hyperlink{duxe9tecteur-photouxe9lectrique-barrage-reflexproximituxe9-1}{2.4.7
Détecteur photoélectrique barrage, reflex/proximité 12}

\protect\hyperlink{capteur-uxe0-effet-hall-simple}{2.4.8 Capteur à effet
Hall simple 12}

\protect\hyperlink{crituxe8res-de-choix-des-duxe9tecteurs}{2.4.9
Critères de choix des détecteurs 13}

\protect\hyperlink{les-capteurs-analogiques}{2.5 Les capteurs
analogiques 14}

\protect\hyperlink{mesures-des-longueurs-et-des-angles-potentiomuxe8tre-linuxe9aire-et-angulaires}{2.5.1
Mesures des longueurs et des angles -- Potentiomètre linéaire et
angulaires 14}

\protect\hyperlink{mesure-de-vitesse-guxe9nuxe9ratrice-tachymuxe9trique}{2.5.2
Mesure de vitesse -- Génératrice tachymétrique 14}

\protect\hyperlink{mesure-de-force-et-de-couple-jauges-de-contraintes-extenso-muxe9triques}{2.5.3
Mesure de force et de couple -- Jauges de contraintes (extenso
métriques) 15}

\protect\hyperlink{mesure-de-force-capteur-piuxe9zo-uxe9lectrique}{2.5.4
Mesure de force -- Capteur piézo électrique 15}

\protect\hyperlink{mesure-de-tempuxe9rature-thermocouple}{2.5.5 Mesure
de température -- Thermocouple 15}

\protect\hyperlink{mesure-de-laccuxe9luxe9ration}{2.5.6 Mesure de
l'accélération 16}

\protect\hyperlink{les-capteurs-numuxe9riques}{2.6 Les capteurs
numériques 16}

\protect\hyperlink{mesure-de-position-et-de-vitesse-codeur-incruxe9mental}{2.6.1
Mesure de position (et de vitesse) -- Codeur incrémental 16}

\protect\hyperlink{mesure-de-position-codeur-absolu}{2.6.2 Mesure de
position -- Codeur absolu 17}

\protect\hyperlink{capteurs-intelligents}{2.7 Capteurs intelligents 17}

\protect\hyperlink{capteur-de-champ-magnuxe9tique-et-dangle-de-rotation-uxe0-effet-hall}{2.7.1
Capteur de champ magnétique et d'angle de rotation à effet Hall 17}

\protect\hyperlink{gyroscopegyromuxe8tre-numuxe9rique}{2.7.2
Gyroscope/gyromètre numérique 17}

\protect\hyperlink{capteur-dattitude-centrale-inertielle-motion-processing-unit.}{2.7.3
Capteur d'attitude / centrale inertielle / M\emph{otion Processing
Unit.} 18}

\protect\hyperlink{autres-capteurs}{2.8 Autres capteurs 18}

\protect\hyperlink{capteurs-de-pression}{2.8.1 Capteurs de pression 18}

\protect\hyperlink{capteur-de-duxe9bit}{2.8.2 Capteur de débit 18}

\protect\hyperlink{capteurs-uxe0-ultrason-sonar-sg}{2.8.3 Capteurs à
ultrason / sonar (SG) 18}

\protect\hyperlink{ruxe9solveur-sg}{2.8.4 Résolveur (SG) 18}

\protect\hyperlink{lidar}{2.8.5 LIDAR 18}

\protect\hyperlink{traitement-de-linformation}{3 Traitement de
l'information 19}

\protect\hyperlink{automate-programmables-industriels-api}{3.1 Automate
programmables industriels (API) 19}

\protect\hyperlink{cartes-programmables-carte-arduino}{3.2 Cartes
programmables -- Carte Arduino 20}

\protect\hyperlink{traitement-avec-cuxe2blage-uxe9lectrique-pneumatique-et-hydraulique}{3.2.1
Traitement avec câblage électrique, pneumatique et hydraulique 20}

\protect\hyperlink{traitement-avec-un-automate-logique}{3.2.2 Traitement
avec un automate logique 20}

\protect\hyperlink{traitement-avec-un-fpga}{3.2.3 Traitement avec un
FPGA 20}

\protect\hyperlink{traitement-avec-un-digital-signal-processor-dsp}{3.2.4
Traitement avec un Digital Signal Processor DSP 20}

\protect\hyperlink{fonction-convertir}{4 Fonction Convertir 21}

\protect\hyperlink{les-moteurs-uxe9lectriques}{4.1 Les moteurs
électriques 21}

\protect\hyperlink{le-moteur-uxe0-courant-continu}{4.1.1 Le moteur à
courant continu 21}

\protect\hyperlink{le-moteur-brushless-sans-broches}{4.1.2 Le moteur
brushless (sans broches) 21}

\protect\hyperlink{le-moteur-asynchrone}{4.1.3 Le moteur asynchrone 21}

\protect\hyperlink{le-moteur-pas-uxe0-pas}{4.1.4 Le moteur pas à pas 22}

\protect\hyperlink{actionneurs-linuxe9aires}{4.1.5 Actionneurs linéaires
22}

\protect\hyperlink{convertisseurs-duxe9nergie-pneumatique-ou-hydraulique}{4.2
Convertisseurs d'énergie pneumatique ou hydraulique 23}

\protect\hyperlink{moteurs-thermiques}{4.3 Moteurs thermiques 24}

\protect\hyperlink{fonction-transmettre}{5 Fonction Transmettre 24}

\protect\hyperlink{eluxe9ments-de-liaisons}{5.1 Eléments de liaisons 24}

\protect\hyperlink{transmission-par-engrenages}{5.2 Transmission par
engrenages 24}

\protect\hyperlink{engrenage-uxe0-contact-extuxe9rieur}{5.2.1 Engrenage
à contact extérieur 24}

\protect\hyperlink{engrenage-uxe0-contact-intuxe9rieur}{5.2.2 Engrenage
à contact intérieur 25}

\protect\hyperlink{roues-et-vis-sans-fin}{5.2.3 Roues et vis sans fin
25}

\protect\hyperlink{renvoi-dangle}{5.2.4 Renvoi d'angle~: 25}

\protect\hyperlink{transmission-par-courroies-et-chaines}{5.3
Transmission par courroies et chaines 25}

\protect\hyperlink{boites-de-vitesse}{5.4 Boites de vitesse 26}

\protect\hyperlink{convertisseurs-multiples}{5.5 Convertisseurs
multiples 26}

\protect\hyperlink{fonction-modulercommuter}{6 Fonction moduler/commuter
26}

\protect\hyperlink{introduction}{6.1 Introduction 26}

\protect\hyperlink{les-modulateurs-uxe9lectriques}{6.2 Les modulateurs
électriques 27}

\protect\hyperlink{le-relai-ou-contacteur-de-puissance}{6.2.1 Le relai
(ou contacteur de puissance) 27}

\protect\hyperlink{le-hacheur-convertisseur-statique}{6.2.2 Le hacheur
(convertisseur statique) 28}

\protect\hyperlink{londuleur-variateur}{6.2.3 L'onduleur (variateur) 29}

\protect\hyperlink{notion-de-schuxe9ma-uxe9lectrique}{6.2.4 Notion de
schéma électrique 29}

\protect\hyperlink{les-modulateurs-pneumatiques-et-hydrauliques}{6.3 Les
modulateurs pneumatiques et hydrauliques 30}

\protect\hyperlink{les-distributeurs}{6.4 Les distributeurs 30}

\protect\hyperlink{duxe9signation-des-distributeurs}{6.5 Désignation des
distributeurs 31}

\protect\hyperlink{fonction-stocker}{7 Fonction stocker 32}

\protect\hyperlink{piles-et-batteries}{7.1 Piles et batteries 32}

\protect\hyperlink{energies-pneumatiques-et-hydrauliques}{7.2 Energies
pneumatiques et hydrauliques 32}

\protect\hyperlink{stockage-par-gravituxe9}{7.3 Stockage par gravité 33}

\protect\hyperlink{stockage-inertiel}{7.4 Stockage inertiel 33}

\protect\hyperlink{ressorts}{7.5 Ressorts 34}

\protect\hyperlink{energie-thermique}{7.6 Energie thermique 34}

\protect\hyperlink{fonction-alimenter}{8 Fonction Alimenter 34}

\protect\hyperlink{alimentation-uxe9lectrique}{8.1 Alimentation
électrique 34}

\protect\hyperlink{transformateur}{8.1.1 Transformateur 34}

\protect\hyperlink{alimentation-uxe9lectrique-uxe0-duxe9coupage}{8.1.2
Alimentation électrique à découpage 34}

\protect\hyperlink{ruxe9gulateur-de-tension}{8.1.3 Régulateur de tension
34}

\protect\hyperlink{alimentation-pneumatique-et-hydraulique}{8.2
Alimentation pneumatique et hydraulique 35}

\protect\hyperlink{ruxe9gulateur-de-pression}{8.2.1 Régulateur de
pression 35}

\protect\hyperlink{systuxe8mes-de-conditionnement}{8.2.2 Systèmes de
conditionnement 35}

\protect\hyperlink{systuxe8mes-de-suxe9curituxe9}{8.2.3 Systèmes de
sécurité 35}

\protect\hyperlink{schuxe9ma-composants-pneumatiques-et-hydrauliques}{9
Schéma -- Composants pneumatiques et hydrauliques 36}

\protect\hyperlink{ressources}{10 Ressources 37}

\hypertarget{fonction-acquuxe9rir}{%
\section{Fonction acquérir}\label{fonction-acquuxe9rir}}

\hypertarget{nature-des-informations-grandeur-analogique-numuxe9rique-binaire}{%
\subsection{\texorpdfstring{Nature des informations \emph{Grandeur
Analogique, Numérique,
Binaire}}{Nature des informations Grandeur Analogique, Numérique, Binaire}}\label{nature-des-informations-grandeur-analogique-numuxe9rique-binaire}}

Dans la chaîne d'information, les informations peuvent être de trois
natures différentes~: analogique, numérique et binaire. Le capteur va
acquérir une grandeur analogique et va la transformer en une tension,
elle-même étant aussi une grandeur analogique. Pour pouvoir être
traitée, elle va d'abord être convertie en information numérique grâce à
un Convertisseur Analogique Numérique (\textbf{CAN}). L'information
numérique est alors traitée et stockée sous forme binaire.

\includegraphics[width=6.43023in,height=2.21159in]{media/image6.png}

\begin{longtable}[]{@{}
  >{\raggedright\arraybackslash}p{(\columnwidth - 0\tabcolsep) * \real{1.00}}@{}}
\toprule
\endhead
\textbf{Définition : Grandeur analogique}

Une information analogique peut prendre, de manière continue, toutes les
valeurs possibles dans un intervalle donné. Un signal analogique peut
être représenté par une courbe continue. \\
\bottomrule
\end{longtable}

\begin{longtable}[]{@{}
  >{\raggedright\arraybackslash}p{(\columnwidth - 0\tabcolsep) * \real{1.00}}@{}}
\toprule
\endhead
\textbf{Exemple :}

Les grandeurs physiques (température, vitesse, position, tension, ...)
sont des grandeurs analogiques. \\
\bottomrule
\end{longtable}

\begin{longtable}[]{@{}
  >{\raggedright\arraybackslash}p{(\columnwidth - 0\tabcolsep) * \real{1.00}}@{}}
\toprule
\endhead
\textbf{Définition : Grandeur numérique}

Une information numérique est constituée de plusieurs bits (variables
binaires 0/1). Elle est en général issue d'un traitement
(échantillonnage et codage) d'une information analogique. On parle de
conversion analogique -- numérique (CAN). \\
\bottomrule
\end{longtable}

\begin{longtable}[]{@{}
  >{\raggedright\arraybackslash}p{(\columnwidth - 0\tabcolsep) * \real{1.00}}@{}}
\toprule
\endhead
\textbf{Définition : Grandeur binaire}

Les informations logiques sont des informations binaires. Elles sont de
type 0 ou 1, vrai ou faux, ouvert ou fermé, tout ou rien (TOR). \\
\bottomrule
\end{longtable}

\begin{longtable}[]{@{}
  >{\raggedright\arraybackslash}p{(\columnwidth - 0\tabcolsep) * \real{1.00}}@{}}
\toprule
\endhead
\textbf{Exemple :}

Variables de type «~boolean~» en Python, état d'un
interrupteur\ldots{} \\
\bottomrule
\end{longtable}

\hypertarget{signal-dans-un-systuxe8me-numuxe9rique-uxe9chantillonnage-et-ruxe9solution}{%
\subsection{signal~dans un système numérique : échantillonnage et
résolution}\label{signal-dans-un-systuxe8me-numuxe9rique-uxe9chantillonnage-et-ruxe9solution}}

Lorsque l'on fait l'acquisition d'un signal de mesure sur un système
numérique, celui-ci doit se conformer à deux contraintes~: la
\textbf{résolution} et \textbf{l'échantillonnage}. La résolution
correspond à l'intervalle minimum entre deux mesures et
l'échantillonnage au temps qui sépare deux prises de mesure. Ces
contraintes font partis des facteurs qui limite la précision et la
connaissance d'une mesure. En fonction de la gamme et du niveau
d'intégration d'un capteur et de son système d'enregistrement ces
paramètres varient fortement par exemple les résolutions usuelles sont
comprises entre 8 et 20 bits soit entre 256 et 1047576 points de mesure.

\emph{Exemple~:} Mesure d'un angle à l'aide d'un potentiomètre
rotatif~sur un microcontrôleur.

\begin{longtable}[]{@{}
  >{\raggedright\arraybackslash}p{(\columnwidth - 2\tabcolsep) * \real{0.50}}
  >{\raggedright\arraybackslash}p{(\columnwidth - 2\tabcolsep) * \real{0.50}}@{}}
\toprule
\begin{minipage}[b]{\linewidth}\raggedright
\begin{enumerate}
\def\labelenumi{\arabic{enumi}.}
\item
  Conversion analogique numérique à l'aide d'un module intégré au
  microcontrôleur. Le convertisseur dispose d'un \emph{nombre de bits}
  permettant de traduire la valeur numériquement. Par exemple, sur un
  contrôleur Arduino Uno 8 bit permettent de coder les signaux
  analogiques. On obtient une résolution de \(2^{8} = 256\) valeurs
  distinctes sur l'intervalle de mesure de 5V. Si le potentiomètre est
  monotour on la résolution est de \(\frac{360}{256} = 1,42{^\circ}\).
  C'est l'intervalle minimum entre 2 mesures.
\end{enumerate}

\begin{quote}
Attention la précision de la mesure est en général moins intéressante
que cet intervalle. Par exemple si le potentiomètre présente un défaut
de linéarité de 2 \% la tension électrique de mesure n'est plus une
mesure parfaite de l'angle avec un défaut de 2\% soit l'équivalent de
7,2°~!
\end{quote}

\begin{enumerate}
\def\labelenumi{\arabic{enumi}.}
\setcounter{enumi}{1}
\item
  Enregistrement de la valeur toute les 1ms (paramétrable dans une
  certaine mesure en fonction de la rapidité du système d'acquisition ou
  de la vitesse d'exécution du programme dans le cas du
  microcontrolleur.
\end{enumerate}
\end{minipage} &
\includegraphics[width=1.23636in,height=1.23636in]{media/image7.jpeg}

\includegraphics[width=1.44713in,height=1in]{media/image8.jpeg} \\
\midrule
\endhead
\includegraphics[width=4.93455in,height=2.32911in]{media/image9.png}
& \\
\bottomrule
\end{longtable}

\hypertarget{caractuxe9ristiques-des-capteurs}{%
\subsection{Caractéristiques des
capteurs}\label{caractuxe9ristiques-des-capteurs}}

\hypertarget{duxe9finitions}{%
\subsubsection{Définitions}\label{duxe9finitions}}

\begin{longtable}[]{@{}l@{}}
\toprule
\endhead
\textbf{Définition : Mesurande --} Grandeur physique à acquérir. \\
\bottomrule
\end{longtable}

\begin{longtable}[]{@{}l@{}}
\toprule
\endhead
\textbf{Définition : Fidélité --} Capacité à donner des résultats
répétables sur un échantillon de mesures. (Figure 1) \\
\bottomrule
\end{longtable}

\begin{longtable}[]{@{}l@{}}
\toprule
\endhead
\textbf{Définition : Justesse --} Capacité à donner une valeur moyenne
correcte sur un échantillon de mesures. (Figure 1) \\
\bottomrule
\end{longtable}

\begin{longtable}[]{@{}l@{}}
\toprule
\endhead
\textbf{Définition : Précision/Exactitude --} Aptitude du capteur à
donner une mesure proche de la valeur vraie. Une combinaison de fidélité
et de justesse permet de l'obtenir (Figure 1). \\
\bottomrule
\end{longtable}

\begin{longtable}[]{@{}l@{}}
\toprule
\endhead
\textbf{Définition : Etalonnage/Calibration --} Opération permettant
d'associer la valeur brute délivrée par un capteur à la mesure
recherchée. Un étalonnage réussi doit permettre d'obtenir des mesures
précise lorsqu'un capteur est fidèle. \\
\bottomrule
\end{longtable}

\begin{longtable}[]{@{}l@{}}
\toprule
\endhead
\textbf{Définition : Étendue de la mesure --} Valeurs extrêmes pouvant
être mesurées par le capteur. \\
\bottomrule
\end{longtable}

\begin{longtable}[]{@{}l@{}}
\toprule
\endhead
\textbf{Définition : Résolution --} Plus petite variation de grandeur
mesurable par le capteur. \\
\bottomrule
\end{longtable}

\begin{longtable}[]{@{}l@{}}
\toprule
\endhead
\textbf{Définition : Sensibilité --} Variation du signal de sortie par
rapport à la variation du signal d'entrée. \\
\bottomrule
\end{longtable}

\begin{longtable}[]{@{}l@{}}
\toprule
\endhead
\textbf{Définition : Rapidité --} Temps de réaction du capteur. La
rapidité est liée à la bande passante du capteur. \\
\bottomrule
\end{longtable}

\includegraphics[width=4.16667in,height=1.66667in]{media/image10.png}

Figure 1 Fidélité Justesse Exactitude

\hypertarget{probluxe8mes-de-mesure-offset-gain-linuxe9arituxe9}{%
\subsubsection{\texorpdfstring{Problèmes de mesure \emph{Offset, Gain,
Linéarité}}{Problèmes de mesure Offset, Gain, Linéarité}}\label{probluxe8mes-de-mesure-offset-gain-linuxe9arituxe9}}

\begin{longtable}[]{@{}lll@{}}
\toprule
\textbf{Erreur de zéro (offset)} & \textbf{Erreur d'échelle (gain)} &
\textbf{Erreur de linéarité} \\
\midrule
\endhead
\emph{Erreur de décalage constant entre la valeur mesurée et la valeur
réelle de la grandeur physique.} & \emph{C'est une erreur qui dépend de
la façon linéaire à la grandeur mesurée.} & \emph{La caractéristique du
capteur n'est pas une droite} \\
\includegraphics[width=1.9685in,height=1.6825in]{media/image11.png} &
\includegraphics[width=1.9685in,height=1.6825in]{media/image12.png} &
\includegraphics[width=1.9685in,height=1.68883in]{media/image13.png} \\
\bottomrule
\end{longtable}

\begin{longtable}[]{@{}ll@{}}
\toprule
\textbf{Erreur d'hystérésis} & \textbf{Erreur de quantification} \\
\midrule
\endhead
\emph{Phénomène apparaissant lorsque le résultat de la mesure dépend de
la précédente mesure.} & \emph{La caractéristique est un «~escalier~».
Cette erreur est due à la conversion analogique -- numérique.} \\
\includegraphics[width=2.16042in,height=1.84653in]{media/image14.png} &
\includegraphics[width=2.16042in,height=1.85347in]{media/image15.png} \\
\bottomrule
\end{longtable}

\hypertarget{les-duxe9tecteurs-capteurs-logiques}{%
\subsection{Les détecteurs -- Capteurs
logiques}\label{les-duxe9tecteurs-capteurs-logiques}}

\begin{longtable}[]{@{}l@{}}
\toprule
\endhead
\textbf{Définition : Détecteurs --} Les détecteurs permettent de
détecter la présence ou l'absence d'un objet ou d'un niveau
prédéterminé. Ils délivrent une information booléenne (vrai/faux) sous
forme électrique, pneumatique ou hydraulique. \\
\bottomrule
\end{longtable}

\hypertarget{duxe9tecteurs-uxe0-contactfin-de-course}{%
\subsubsection{Détecteurs à contact/Fin de
course}\label{duxe9tecteurs-uxe0-contactfin-de-course}}

\begin{longtable}[]{@{}
  >{\raggedright\arraybackslash}p{(\columnwidth - 2\tabcolsep) * \real{0.50}}
  >{\raggedright\arraybackslash}p{(\columnwidth - 2\tabcolsep) * \real{0.50}}@{}}
\toprule
\textbf{Nature de la grandeur détectée~:} Contact

\textbf{Nature du signal délivré~:} Signal électrique

\textbf{Symbole~:} interrupteurs normalement ouverts et fermés.

\includegraphics[width=1.34375in,height=0.875in]{media/image16.png} &
\includegraphics[width=0.82986in,height=1.71944in]{media/image17.png}\textbf{Principe
de fonctionnement}

Ce détecteur est un interrupteur de position permettant de délivrer une
information «~Tout ou rien~» en fonction de la position d'un organe de
commande.

Un tel détecteur est alimenté (par exemple en 5V -- 2 fils). On mesure
alors la tension sur une borne de sortie. (L'état de la tension mesurée
correspond à l'état ouvert ou fermé de l'interrupteur.) \\
\midrule
\endhead
\begin{minipage}[t]{\linewidth}\raggedright
\includegraphics[width=3.23256in,height=1.72406in]{media/image18.png}\includegraphics[width=1.91839in,height=1.61628in]{media/image19.png}

\begin{longtable}[]{@{}l@{}}
\toprule
\textbf{Exemples :} Détecteur de présence de bocal capsuleuse, fin de
course du mors de la cordeuse, \ldots{} \\
\midrule
\endhead
 \\
\bottomrule
\end{longtable}
\end{minipage} & \\
\bottomrule
\end{longtable}

\hypertarget{capteur-pneumatique}{%
\subsubsection{Capteur pneumatique}\label{capteur-pneumatique}}

\begin{longtable}[]{@{}
  >{\raggedright\arraybackslash}p{(\columnwidth - 2\tabcolsep) * \real{0.50}}
  >{\raggedright\arraybackslash}p{(\columnwidth - 2\tabcolsep) * \real{0.50}}@{}}
\toprule
\textbf{Nature de la grandeur détectée~:} Contact

\textbf{Nature du signal délivré~:} Signal pneumatique

\textbf{Symbole pneumatique}

\includegraphics[width=0.70453in,height=1.14134in]{media/image20.png} &
\includegraphics[width=1.85833in,height=1.59236in]{media/image21.png}\textbf{Principe
de fonctionnement}

L'air arrive par l'orifice inférieur. Un orifice est relié à la sortie.
Lorsqu'on presse sur le galet, de l'air peut alors passer par l'orifice
de sortie.

On détecte ainsi la présence d'un objet. \\
\midrule
\endhead
\begin{minipage}[t]{\linewidth}\raggedright
\begin{longtable}[]{@{}
  >{\raggedright\arraybackslash}p{(\columnwidth - 2\tabcolsep) * \real{0.50}}
  >{\raggedright\arraybackslash}p{(\columnwidth - 2\tabcolsep) * \real{0.50}}@{}}
\toprule
&
\includegraphics[width=1.30266in,height=0.93809in]{media/image22.png} \\
\midrule
\endhead
\textbf{Exemples :}

Pas d'exemple dans notre laboratoire. & \\
& \\
\bottomrule
\end{longtable}
\end{minipage} & \\
\bottomrule
\end{longtable}

\hypertarget{interrupteur-uxe0-lame-souples-ils}{%
\subsubsection{Interrupteur à lame souples
(ILS)}\label{interrupteur-uxe0-lame-souples-ils}}

\begin{longtable}[]{@{}
  >{\raggedright\arraybackslash}p{(\columnwidth - 2\tabcolsep) * \real{0.50}}
  >{\raggedright\arraybackslash}p{(\columnwidth - 2\tabcolsep) * \real{0.50}}@{}}
\toprule
\textbf{Nature de la grandeur détectée~:} proximité

\includegraphics[width=0.56667in,height=0.86389in]{media/image23.png}\textbf{Nature
du signal délivré~:} Signal électrique

\textbf{Symbole~:} & \textbf{Principe de fonctionnement}

\includegraphics[width=2.55972in,height=1.5in]{media/image24.png}Les
détecteurs ILS équipent les vérins, permettant de détecter la présence
de la tige aux extrémités du vérin.

Ils sont formés de deux lames métalliques souples très proches l'une de
l'autre. Si le capteur est placé dans un champ magnétique alors les deux
lames souples se mettent en contact et un courant électrique peut
circuler de l'une vers l'autre. \\
\midrule
\endhead
\begin{minipage}[t]{\linewidth}\raggedright
\begin{longtable}[]{@{}
  >{\raggedright\arraybackslash}p{(\columnwidth - 4\tabcolsep) * \real{0.33}}
  >{\raggedright\arraybackslash}p{(\columnwidth - 4\tabcolsep) * \real{0.33}}
  >{\raggedright\arraybackslash}p{(\columnwidth - 4\tabcolsep) * \real{0.33}}@{}}
\toprule
\endhead
\textbf{Exemples :}

Vérins de la capsuleuse. &
\includegraphics[width=1.60157in,height=0.74426in]{media/image25.png} &
\includegraphics[width=1.33636in,height=0.87209in]{media/image26.png} \\
\bottomrule
\end{longtable}
\end{minipage} & \\
\bottomrule
\end{longtable}

\hypertarget{duxe9tecteur-photouxe9lectrique-barrage-reflexproximituxe9}{%
\subsubsection{Détecteur photoélectrique barrage,
reflex/proximité}\label{duxe9tecteur-photouxe9lectrique-barrage-reflexproximituxe9}}

\begin{longtable}[]{@{}
  >{\raggedright\arraybackslash}p{(\columnwidth - 2\tabcolsep) * \real{0.50}}
  >{\raggedright\arraybackslash}p{(\columnwidth - 2\tabcolsep) * \real{0.50}}@{}}
\toprule
\textbf{Nature de la grandeur détectée~:} proximité

\textbf{Nature du signal délivré~:} Signal électrique

\textbf{Symbole~:}

\includegraphics[width=0.45169in,height=0.75541in]{media/image27.png} &
\textbf{Principe de fonctionnement}

Un détecteur photoélectrique est composé d'un émetteur (DEL) et d'un
récepteur (phototransistor). Lorsque émetteur et récepteur sont
dissociés, o parle de barrage. Sinon, on parle de reflex (existante
d'une cible réfléchissante) ou de système proximité.

Dans le cas du barrage ou du reflex, on détecte une pièce lorsque le
faisceau lumineux est coupé. Dans le cas du système proximité, la pièce
réfléchit le faisceau.

\includegraphics[width=3.32014in,height=0.98681in]{media/image28.png} \\
\midrule
\endhead
\begin{minipage}[t]{\linewidth}\raggedright
\begin{longtable}[]{@{}
  >{\raggedright\arraybackslash}p{(\columnwidth - 4\tabcolsep) * \real{0.33}}
  >{\raggedright\arraybackslash}p{(\columnwidth - 4\tabcolsep) * \real{0.33}}
  >{\raggedright\arraybackslash}p{(\columnwidth - 4\tabcolsep) * \real{0.33}}@{}}
\toprule
\endhead
\textbf{Exemples :}

Ces détecteurs peuvent être utilisés dans les codeurs incrémentaux (Voir
plus loin).

Ils permettent de détecter des objets transparents, opaques\ldots{} &
\includegraphics[width=1.11628in,height=1.11628in]{media/image29.jpeg} &
\includegraphics[width=1.00184in,height=1.18405in]{media/image30.jpeg} \\
\bottomrule
\end{longtable}
\end{minipage} & \\
\bottomrule
\end{longtable}

\hypertarget{duxe9tecteur-inductif}{%
\subsubsection{Détecteur inductif}\label{duxe9tecteur-inductif}}

\begin{longtable}[]{@{}
  >{\raggedright\arraybackslash}p{(\columnwidth - 2\tabcolsep) * \real{0.50}}
  >{\raggedright\arraybackslash}p{(\columnwidth - 2\tabcolsep) * \real{0.50}}@{}}
\toprule
\textbf{Nature de la grandeur détectée~:} proximité

\textbf{Nature du signal délivré~:} Signal électrique

\textbf{Symbole~:}

\includegraphics[width=0.52083in,height=0.89583in]{media/image31.png} &
\textbf{Principe de fonctionnement}

\includegraphics[width=2.07153in,height=0.79167in]{media/image32.png}Ces
détecteurs sont utilisés pour détecter la présence, l'absence ou le
passage d'un \textbf{objet métallique}. Les capteurs inductifs
produisent à l'extrémité leur tête de détection un champ magnétique
oscillant. Ce champ est généré par une inductance et une capacité montée
en parallèle. Lorsqu'un \textbf{objet métallique} pénètre dans ce champ,
il y a perturbation de ce champ puis atténuation du champ magnétique
oscillant. Cela provoque ainsi le changement d'état de sortie du
détecteur (passage de l'état 0 à l'état 1). \\
\midrule
\endhead
\begin{minipage}[t]{\linewidth}\raggedright
\begin{longtable}[]{@{}
  >{\raggedright\arraybackslash}p{(\columnwidth - 4\tabcolsep) * \real{0.33}}
  >{\raggedright\arraybackslash}p{(\columnwidth - 4\tabcolsep) * \real{0.33}}
  >{\raggedright\arraybackslash}p{(\columnwidth - 4\tabcolsep) * \real{0.33}}@{}}
\toprule
& \includegraphics[width=1.81395in,height=1.59981in]{media/image33.png}
&
\includegraphics[width=1.0391in,height=0.93592in]{media/image34.png} \\
\midrule
\endhead
\textbf{Exemples :}

Sur la capsuleuse ils permettent de détecter l'état de serrage sur la
capsule ou la présence du maneton avant que celui-ci n'entre dans la
croix de Malte. & & \\
& & \\
\bottomrule
\end{longtable}
\end{minipage} & \\
\bottomrule
\end{longtable}

\hypertarget{duxe9tecteur-capacitif}{%
\subsubsection{Détecteur capacitif}\label{duxe9tecteur-capacitif}}

\begin{longtable}[]{@{}
  >{\raggedright\arraybackslash}p{(\columnwidth - 2\tabcolsep) * \real{0.50}}
  >{\raggedright\arraybackslash}p{(\columnwidth - 2\tabcolsep) * \real{0.50}}@{}}
\toprule
\includegraphics[width=0.6875in,height=1.15625in]{media/image35.png}\textbf{Nature
de la grandeur détectée~:} proximité

\textbf{Nature du signal délivré~:}

Signal électrique

\textbf{Symbole~:} & \textbf{Principe de fonctionnement}

Ces capteurs permettent de détecter \textbf{tous types de matériaux}.
Lorsqu'un objet est à proximité du détecteur, il perturbe le champ
électrique entre les deux électrodes. \\
\midrule
\endhead
\begin{minipage}[t]{\linewidth}\raggedright
\begin{longtable}[]{@{}
  >{\raggedright\arraybackslash}p{(\columnwidth - 2\tabcolsep) * \real{0.50}}
  >{\raggedright\arraybackslash}p{(\columnwidth - 2\tabcolsep) * \real{0.50}}@{}}
\toprule
\endhead
\textbf{Exemples :}

Absents sur nos systèmes de laboratoire. Ils sont utilisés lorsque les
détecteurs inductifs ne peuvent pas être utilisés.

La distance de détection est très faible. &
\includegraphics[width=2.08199in,height=1.61628in]{media/image36.png} \\
\bottomrule
\end{longtable}
\end{minipage} & \\
\bottomrule
\end{longtable}

\hypertarget{duxe9tecteur-photouxe9lectrique-barrage-reflexproximituxe9-1}{%
\subsubsection{Détecteur photoélectrique barrage,
reflex/proximité}\label{duxe9tecteur-photouxe9lectrique-barrage-reflexproximituxe9-1}}

\begin{longtable}[]{@{}
  >{\raggedright\arraybackslash}p{(\columnwidth - 2\tabcolsep) * \real{0.50}}
  >{\raggedright\arraybackslash}p{(\columnwidth - 2\tabcolsep) * \real{0.50}}@{}}
\toprule
\textbf{Nature de la grandeur détectée~:} proximité

\textbf{Nature du signal délivré~:} Signal électrique

\textbf{Symbole~:}

\includegraphics[width=0.45169in,height=0.75541in]{media/image27.png} &
\textbf{Principe de fonctionnement}

Un détecteur photoélectrique est composé d'un émetteur (DEL) et d'un
récepteur (phototransistor). Lorsque émetteur et récepteur sont
dissociés, o parle de barrage. Sinon, on parle de reflex (existante
d'une cible réfléchissante) ou de système proximité.

Dans le cas du barrage ou du reflex, on détecte une pièce lorsque le
faisceau lumineux est coupé. Dans le cas du système proximité, la pièce
réfléchit le faisceau.

\includegraphics[width=3.32014in,height=0.98681in]{media/image28.png} \\
\midrule
\endhead
\begin{minipage}[t]{\linewidth}\raggedright
\begin{longtable}[]{@{}
  >{\raggedright\arraybackslash}p{(\columnwidth - 4\tabcolsep) * \real{0.33}}
  >{\raggedright\arraybackslash}p{(\columnwidth - 4\tabcolsep) * \real{0.33}}
  >{\raggedright\arraybackslash}p{(\columnwidth - 4\tabcolsep) * \real{0.33}}@{}}
\toprule
\endhead
\textbf{Exemples :}

Ces détecteurs peuvent être utilisés dans les codeurs incrémentaux (Voir
plus loin).

Ils permettent de détecter des objets transparents, opaques\ldots{} &
\includegraphics[width=1.11628in,height=1.11628in]{media/image29.jpeg} &
\includegraphics[width=1.00184in,height=1.18405in]{media/image30.jpeg} \\
\bottomrule
\end{longtable}
\end{minipage} & \\
\bottomrule
\end{longtable}

\hypertarget{capteur-uxe0-effet-hall-simple}{%
\subsubsection{Capteur à effet Hall
simple}\label{capteur-uxe0-effet-hall-simple}}

\begin{longtable}[]{@{}
  >{\raggedright\arraybackslash}p{(\columnwidth - 0\tabcolsep) * \real{1.00}}@{}}
\toprule
\endhead
\includegraphics[width=5.26603in,height=3.61694in]{media/image37.jpeg}\textbf{Grandeurs
détectées~:}

Présence d'un champ magnétique, d'un aimant, angle ou vitesse de
rotation à l'aide d'un disque à franges.

\textbf{Exemple~:}

Capteur point mort haut moteur voiture.~

\textbf{~}

Ce capteur utilise l'effet d'induction afin de détecter la présence d'un
champ magnétique. Il permet en général de détecter le passage d'un
aimant. C'est un capteur de présence sans contact très fiable. Par un
montage adapté une vitesse ou un angle de rotation peuvent être mesurés
par comptage. \\
\bottomrule
\end{longtable}

\hypertarget{crituxe8res-de-choix-des-duxe9tecteurs}{%
\subsubsection{Critères de choix des
détecteurs}\label{crituxe8res-de-choix-des-duxe9tecteurs}}

\includegraphics[width=6.34571in,height=4.90909in]{media/image38.png}

\hypertarget{les-capteurs-analogiques}{%
\subsection{Les capteurs analogiques}\label{les-capteurs-analogiques}}

Ces capteurs permettent de mesurer une grandeur physique. Ils délivrent
un signal continu.

\hypertarget{mesures-des-longueurs-et-des-angles-potentiomuxe8tre-linuxe9aire-et-angulaires}{%
\subsubsection{Mesures des longueurs et des angles -- Potentiomètre
linéaire et
angulaires}\label{mesures-des-longueurs-et-des-angles-potentiomuxe8tre-linuxe9aire-et-angulaires}}

\begin{longtable}[]{@{}
  >{\raggedright\arraybackslash}p{(\columnwidth - 2\tabcolsep) * \real{0.50}}
  >{\raggedright\arraybackslash}p{(\columnwidth - 2\tabcolsep) * \real{0.50}}@{}}
\toprule
\textbf{Nature de la grandeur détectée~:} angle ou distance

\textbf{Nature du signal délivré~:}

Signal électrique

\includegraphics[width=2.42025in,height=1.80297in]{media/image39.png} &
\textbf{Principe de fonctionnement~:} Ces capteurs fonctionnent comme un
rhéostat~: un curseur se déplace sur une piste (linéaire ou circulaire).
Un pont diviseur de tension permet de déterminer la tension. Connaissant
la course du capteur, on peut en déduire la correspondance entre tension
et dimension. En plaçant une tension +Vcc en B et une masse en A, la
tension mesurée entre C et la masse sera comprise entre 0V et +Vcc et
proportionnelle à la grandeur mécanique que l'on souhaite mesurer (angle
ou distance).

\includegraphics[width=3.58333in,height=1.59514in]{media/image40.jpeg} \\
\midrule
\endhead
\begin{minipage}[t]{\linewidth}\raggedright
\begin{longtable}[]{@{}
  >{\raggedright\arraybackslash}p{(\columnwidth - 4\tabcolsep) * \real{0.33}}
  >{\raggedright\arraybackslash}p{(\columnwidth - 4\tabcolsep) * \real{0.33}}
  >{\raggedright\arraybackslash}p{(\columnwidth - 4\tabcolsep) * \real{0.33}}@{}}
\toprule
\textbf{Exemples :}

Position angulaire du bras du MaxPID, position angulaire des volants de
la DAE et de la DIRAVI, position angulaire des roues de la DAE, mesure
de l'écrasement du ressort de la cordeuse, position angulaire des
ventaux du portail\ldots{}

On peut remarquer qu'un potentiomètre comporte 3 fils (alimentation,
masse et mesure). &
\includegraphics[width=1.44733in,height=1.05222in]{media/image41.png} &
\includegraphics[width=1.63636in,height=1.38654in]{media/image42.png} \\
\midrule
\endhead
& & \\
\bottomrule
\end{longtable}
\end{minipage} & \\
\bottomrule
\end{longtable}

\hypertarget{mesure-de-vitesse-guxe9nuxe9ratrice-tachymuxe9trique}{%
\subsubsection{Mesure de vitesse -- Génératrice
tachymétrique}\label{mesure-de-vitesse-guxe9nuxe9ratrice-tachymuxe9trique}}

\begin{longtable}[]{@{}
  >{\raggedright\arraybackslash}p{(\columnwidth - 2\tabcolsep) * \real{0.50}}
  >{\raggedright\arraybackslash}p{(\columnwidth - 2\tabcolsep) * \real{0.50}}@{}}
\toprule
\textbf{Nature de la grandeur détectée~:} vitesse

\textbf{Nature du signal délivré~:}

Signal électrique & \textbf{Principe de fonctionnement}

Une génératrice tachymétrique a la même structure qu'un moteur à courant
continu. Lorsque l'axe du va tourner, il va générer une tension
proportionnelle à sa fréquence de rotation. \\
\midrule
\endhead
\begin{minipage}[t]{\linewidth}\raggedright
\begin{longtable}[]{@{}
  >{\raggedright\arraybackslash}p{(\columnwidth - 2\tabcolsep) * \real{0.50}}
  >{\raggedright\arraybackslash}p{(\columnwidth - 2\tabcolsep) * \real{0.50}}@{}}
\toprule
&
\includegraphics[width=2.51948in,height=1.44156in]{media/image43.jpeg} \\
\midrule
\endhead
\textbf{Exemples :}

Mesure de la vitesse du moteur du MaxPID ou des vérins électriques de la
plateforme 6 axes. & \\
& \\
\bottomrule
\end{longtable}
\end{minipage} & \\
\bottomrule
\end{longtable}

\hypertarget{mesure-de-force-et-de-couple-jauges-de-contraintes-extenso-muxe9triques}{%
\subsubsection{Mesure de force et de couple -- Jauges de contraintes
(extenso
métriques)}\label{mesure-de-force-et-de-couple-jauges-de-contraintes-extenso-muxe9triques}}

\begin{longtable}[]{@{}
  >{\raggedright\arraybackslash}p{(\columnwidth - 2\tabcolsep) * \real{0.50}}
  >{\raggedright\arraybackslash}p{(\columnwidth - 2\tabcolsep) * \real{0.50}}@{}}
\toprule
\textbf{Nature de la grandeur détectée~:} effort ou couple

\textbf{Nature du signal délivré~:} Signal électrique

\includegraphics[width=0.89583in,height=1.16875in]{media/image44.jpeg} &
\textbf{Principe de fonctionnement}

Un capteur d'effort est constitué d'un corps d'épreuve, déformable, sur
lequel est collée une jauge. La jauge est constituée d'un fil réalisant
des «~aller-retour~» (cf image). Lorsque le corps d'épreuve va être
soumis à un effort, il va se déformer. Les fils vont alors s'allonger ou
se rétracter, changeant ainsi sa résistance.

La variation de résistance est proportionnelle à l'effort auquel est
soumis le corps d'épreuve. La variation de résistance se mesure par une
variation de tension mesurée elle-même par un pont de Wheatstone.

\(F = E\varepsilon\) et \(\varepsilon = \frac{\Delta L}{L}\) (effort
proportionnel à la déformation), \(\frac{\text{δR}}{R} = K\varepsilon\)
(différentiel de résistance proportionnel à la déformation). \\
\midrule
\endhead
\begin{minipage}[t]{\linewidth}\raggedright
\begin{longtable}[]{@{}
  >{\raggedright\arraybackslash}p{(\columnwidth - 2\tabcolsep) * \real{0.50}}
  >{\raggedright\arraybackslash}p{(\columnwidth - 2\tabcolsep) * \real{0.50}}@{}}
\toprule
&
\includegraphics[width=1.03306in,height=0.96104in]{media/image45.png} \\
\midrule
\endhead
\textbf{Exemples :}

Mesure de l'effort dans le portail, capteur d'effort relié à la corde
sur la cordeuse\ldots{} & \\
& \\
\bottomrule
\end{longtable}
\end{minipage} & \\
\bottomrule
\end{longtable}

\hypertarget{mesure-de-force-capteur-piuxe9zo-uxe9lectrique}{%
\subsubsection{Mesure de force -- Capteur piézo
électrique}\label{mesure-de-force-capteur-piuxe9zo-uxe9lectrique}}

\begin{longtable}[]{@{}
  >{\raggedright\arraybackslash}p{(\columnwidth - 2\tabcolsep) * \real{0.50}}
  >{\raggedright\arraybackslash}p{(\columnwidth - 2\tabcolsep) * \real{0.50}}@{}}
\toprule
\textbf{Nature de la grandeur détectée~:} effort

\textbf{Nature du signal délivré~:} Signal électrique & \textbf{Principe
de fonctionnement}

Les matériaux piézoélectriques ont la propriété de se polariser sous
l'action d'une contrainte mécanique \\
\midrule
\endhead
\begin{minipage}[t]{\linewidth}\raggedright
\begin{longtable}[]{@{}
  >{\raggedright\arraybackslash}p{(\columnwidth - 4\tabcolsep) * \real{0.33}}
  >{\raggedright\arraybackslash}p{(\columnwidth - 4\tabcolsep) * \real{0.33}}
  >{\raggedright\arraybackslash}p{(\columnwidth - 4\tabcolsep) * \real{0.33}}@{}}
\toprule
& &
\includegraphics[width=1.94962in,height=1.10498in]{media/image46.png} \\
\midrule
\endhead
\textbf{Exemples :}

Ces capteurs peuvent être utilisés dans plusieurs autres capteurs~:
capteurs d'efforts, d'accélération\ldots{} &
\includegraphics[width=1.34341in,height=0.88312in]{media/image47.png}
& \\
& & \\
\bottomrule
\end{longtable}
\end{minipage} & \\
\bottomrule
\end{longtable}

\hypertarget{mesure-de-tempuxe9rature-thermocouple}{%
\subsubsection{Mesure de température --
Thermocouple}\label{mesure-de-tempuxe9rature-thermocouple}}

\begin{longtable}[]{@{}
  >{\raggedright\arraybackslash}p{(\columnwidth - 2\tabcolsep) * \real{0.50}}
  >{\raggedright\arraybackslash}p{(\columnwidth - 2\tabcolsep) * \real{0.50}}@{}}
\toprule
\textbf{Nature de la grandeur détectée~:} température

\textbf{Nature du signal délivré~:} Signal électrique & \textbf{Principe
de fonctionnement}

Un thermocouple est constitué de deux fils de matériaux différents
reliés entre eux. Sous l'effet d'un changement de température, on mesure
peut alors mesurer une différence de potentiel entre les fils. \\
\midrule
\endhead
\begin{minipage}[t]{\linewidth}\raggedright
\begin{longtable}[]{@{}
  >{\raggedright\arraybackslash}p{(\columnwidth - 2\tabcolsep) * \real{0.50}}
  >{\raggedright\arraybackslash}p{(\columnwidth - 2\tabcolsep) * \real{0.50}}@{}}
\toprule
&
\includegraphics[width=1.25282in,height=1.26042in]{media/image48.png} \\
\midrule
\endhead
\textbf{Exemples :}

Absent dans nos systèmes de laboratoire. & \\
& \\
\bottomrule
\end{longtable}
\end{minipage} & \\
\bottomrule
\end{longtable}

\hypertarget{mesure-de-laccuxe9luxe9ration}{%
\subsubsection{Mesure de
l'accélération}\label{mesure-de-laccuxe9luxe9ration}}

\begin{longtable}[]{@{}
  >{\raggedright\arraybackslash}p{(\columnwidth - 2\tabcolsep) * \real{0.50}}
  >{\raggedright\arraybackslash}p{(\columnwidth - 2\tabcolsep) * \real{0.50}}@{}}
\toprule
\textbf{Nature de la grandeur détectée~:} tension électrique

\textbf{Nature du signal délivré~:} Signal électrique & \textbf{Principe
de fonctionnement}

Un accéléromètre est un dispositif destiné à mesurer l'accélération. Il
est typiquement constitué de deux éléments~: une masse et un capteur

L'accéléromètre utilisé est à détection piézoélectrique à compression.

\includegraphics[width=2.50903in,height=1.95903in]{media/image49.png}La
tension de sortie Vs est proportionnelle à la charge exercée par le
ressort et la masse sur les disques D. \\
\midrule
\endhead
\begin{minipage}[t]{\linewidth}\raggedright
\begin{longtable}[]{@{}
  >{\raggedright\arraybackslash}p{(\columnwidth - 2\tabcolsep) * \real{0.50}}
  >{\raggedright\arraybackslash}p{(\columnwidth - 2\tabcolsep) * \real{0.50}}@{}}
\toprule
\textbf{Exemples :}

Suspension de VTT didactisée.

Les systèmes de grandes diffusion (drone, smartphone) utilisent des
accéléromètre miniaturisés gravés sur du silicium. On parle de capteur
MEMS. &
\includegraphics[width=1.89219in,height=1.25917in]{media/image50.png} \\
\midrule
\endhead
& \\
\bottomrule
\end{longtable}
\end{minipage} & \\
\bottomrule
\end{longtable}

\hypertarget{les-capteurs-numuxe9riques}{%
\subsection{Les capteurs numériques}\label{les-capteurs-numuxe9riques}}

Ces capteurs permettent de mesurer une grandeur physique. Ils délivrent
un signal échantillonné.

\hypertarget{mesure-de-position-et-de-vitesse-codeur-incruxe9mental}{%
\subsubsection{Mesure de position (et de vitesse) -- Codeur
incrémental}\label{mesure-de-position-et-de-vitesse-codeur-incruxe9mental}}

\begin{longtable}[]{@{}
  >{\raggedright\arraybackslash}p{(\columnwidth - 2\tabcolsep) * \real{0.50}}
  >{\raggedright\arraybackslash}p{(\columnwidth - 2\tabcolsep) * \real{0.50}}@{}}
\toprule
\textbf{Nature de la grandeur détectée~:} proximité

\textbf{Nature du signal délivré~:} Signal électrique &
\begin{minipage}[b]{\linewidth}\raggedright
\includegraphics[width=1.54167in,height=1.51667in]{media/image51.png}\textbf{Principe
de fonctionnement}

Un codeur absolu est composé d'un disque comportant~:

\begin{itemize}
\item
  \begin{quote}
  une piste composée de fentes espacés régulièrements sur sa
  périphérie~;
  \end{quote}
\item
  \begin{quote}
  une seconde piste composée d'une seule fente permettant de faire une
  remise à zéro~;
  \end{quote}
\item
  \begin{quote}
  3 couples diode/photorésistances (ou technologie équivalente)~:
  \end{quote}

  \begin{itemize}
  \item
    \begin{quote}
    \includegraphics[width=2.18333in,height=1.15417in]{media/image52.png}deux
    repérant les fentes sur la périphérie (décalées d'un quart de
    fente)~;
    \end{quote}
  \item
    \begin{quote}
    une repérant la fente de la seconde piste.
    \end{quote}
  \end{itemize}
\end{itemize}

En détectant les fentes sur la piste extérieure, il est possible de
détecter la position angulaire et le sens de rotation.

La piste intérieure facultative permet d'identifier une référence.
\end{minipage} \\
\midrule
\endhead
\begin{minipage}[t]{\linewidth}\raggedright
\begin{longtable}[]{@{}
  >{\raggedright\arraybackslash}p{(\columnwidth - 2\tabcolsep) * \real{0.50}}
  >{\raggedright\arraybackslash}p{(\columnwidth - 2\tabcolsep) * \real{0.50}}@{}}
\toprule
& \includegraphics[width=1.67857in,height=1.25in]{media/image53.png} \\
\midrule
\endhead
\textbf{Exemples :}

Axe numérique, boîte de vitesse robotisée, axes de déplacement des
machines-outils\ldots{} La résolution angulaire du capteur dépend du
nombre de fentes~: \(\frac{360{^\circ}}{n}\). & \\
& \\
\bottomrule
\end{longtable}
\end{minipage} & \\
\bottomrule
\end{longtable}

\hypertarget{mesure-de-position-codeur-absolu}{%
\subsubsection{Mesure de position -- Codeur
absolu}\label{mesure-de-position-codeur-absolu}}

\begin{longtable}[]{@{}
  >{\raggedright\arraybackslash}p{(\columnwidth - 2\tabcolsep) * \real{0.50}}
  >{\raggedright\arraybackslash}p{(\columnwidth - 2\tabcolsep) * \real{0.50}}@{}}
\toprule
\textbf{Nature de la grandeur détectée~:} proximité

\textbf{Nature du signal délivré~:} Signal électrique

\textbf{Symbole~:} & \textbf{Principe de fonctionnement}

\includegraphics[width=1.66667in,height=1.16458in]{media/image54.png}Un
codeur absolu est composé d'un disque de \(n\) pistes. Les pistes
présentes des fentes ou de la matière disposées selon le codage gray
(binaire réfléchi). Une photorésistance permet d'identifier une séquence
de fentes et. Cette séquence correspond à la position angulaire du
disque. \\
\midrule
\endhead
\begin{minipage}[t]{\linewidth}\raggedright
\begin{longtable}[]{@{}
  >{\raggedright\arraybackslash}p{(\columnwidth - 4\tabcolsep) * \real{0.33}}
  >{\raggedright\arraybackslash}p{(\columnwidth - 4\tabcolsep) * \real{0.33}}
  >{\raggedright\arraybackslash}p{(\columnwidth - 4\tabcolsep) * \real{0.33}}@{}}
\toprule
\textbf{Exemples :}

Installation de sureté ou la mise hors tension ne doit pas entrainer une
prise d'origine \emph{Réacteur nucléaire} &
\includegraphics[width=2.14693in,height=1.375in]{media/image55.png} &
\includegraphics[width=1.25023in,height=1.07292in]{media/image56.png} \\
\midrule
\endhead
& & \\
\bottomrule
\end{longtable}
\end{minipage} & \\
\bottomrule
\end{longtable}

\hypertarget{capteurs-intelligents}{%
\subsection{Capteurs intelligents}\label{capteurs-intelligents}}

Ces capteurs modernes profitent des progrès de miniaturisation et dans
le traitement de l'information pour mesurer des grandeurs plus
efficacement.

\hypertarget{capteur-de-champ-magnuxe9tique-et-dangle-de-rotation-uxe0-effet-hall}{%
\subsubsection{Capteur de champ magnétique et d'angle de rotation à
effet
Hall}\label{capteur-de-champ-magnuxe9tique-et-dangle-de-rotation-uxe0-effet-hall}}

\begin{longtable}[]{@{}
  >{\raggedright\arraybackslash}p{(\columnwidth - 2\tabcolsep) * \real{0.50}}
  >{\raggedright\arraybackslash}p{(\columnwidth - 2\tabcolsep) * \real{0.50}}@{}}
\toprule
\endhead
\includegraphics[width=2.60208in,height=2.49792in]{media/image57.jpeg} &
Une grille de capteurs à effet Hall est utilisée afin de mesurer
l'orientation et l'intensité d'un champ magnétique. Ils peuvent ainsi
servir de boussole 3D. Associé à un aimant ils permettent de mesurer
avec précision et de manière absolue un angle sans contact.

\emph{Exemple~boussole : Drone}

\emph{Exemple mesure d'angle~: Cheville du robot NAO.}

Une unité de traitement de l'information est intégrée pour conditionner
et traiter l'information des sous-capteurs de la grille et fournir des
données facilement exploitables. Ces capteurs sont miniaturisés et
mesurent quelque mm². \\
\bottomrule
\end{longtable}

\hypertarget{gyroscopegyromuxe8tre-numuxe9rique}{%
\subsubsection{Gyroscope/gyromètre
numérique}\label{gyroscopegyromuxe8tre-numuxe9rique}}

\begin{longtable}[]{@{}
  >{\raggedright\arraybackslash}p{(\columnwidth - 2\tabcolsep) * \real{0.50}}
  >{\raggedright\arraybackslash}p{(\columnwidth - 2\tabcolsep) * \real{0.50}}@{}}
\toprule
\endhead
\includegraphics[width=2.69514in,height=2.02222in]{media/image58.jpeg} &
Ces capteurs miniaturisés MEMS permettent de mesurer des vitesses de
rotation sans contact avec la référence (repère galiléen), ce qui est
indispensable pour les véhicules par exemple. Un barreau oscille à
grande vitesse, lorsque le support tourne les effets d'inertie génèrent
des forces proportionnelles à la vitesse de rotation. Ainsi mesure ces
forces permet d'obtenir une image de cette vitesse.

Il existe aussi des modèles plus traditionnels qui utilisent un disque
tournant à grande vitesse dont les angle restent fixe par effet
d'inertie. Ces modèles présentent l'avantage de pouvoir mesurer
directement les angles et non les vitesses de rotation. Ils présentent
l'inconvénient d'être beaucoup plus onéreux et emcombrant. \\
\bottomrule
\end{longtable}

\hypertarget{capteur-dattitude-centrale-inertielle-motion-processing-unit.}{%
\subsubsection{\texorpdfstring{Capteur d'attitude / centrale inertielle
/ M\emph{otion Processing
Unit.}}{Capteur d'attitude / centrale inertielle / Motion Processing Unit.}}\label{capteur-dattitude-centrale-inertielle-motion-processing-unit.}}

\begin{longtable}[]{@{}
  >{\raggedright\arraybackslash}p{(\columnwidth - 2\tabcolsep) * \real{0.50}}
  >{\raggedright\arraybackslash}p{(\columnwidth - 2\tabcolsep) * \real{0.50}}@{}}
\toprule
\endhead
\includegraphics[width=2.00823in,height=1.43494in]{media/image59.png} &
En combinant les capteurs miniaturisés MEMS \emph{gyroscope,
accéléromètre} et \emph{boussole} il est possible d'obtenir les angles
d'orientation dans l'espace d'un objet sans système de mesure associé à
la référence (caméra\ldots) et sans contact.

L'accéléromètre mesure en permanence l'accélération verticale, ce qui
donne, en moyenne une orientation fiable de l'axe vertical.

La boussole donne en moyenne une bonne direction du nord magnétique,
permettant de s'orienter dans le plan horizontal.

Le gyroscope numérique donne de bonnes mesures instantanées des angles
et des vitesses de rotation, en général sa dérive empêche une mesure
d'angle fiable dans le temps.

Par combinaison des informations de ces 3 capteurs, par exemple avec les
filtres de Kalmann, on peut obtenir une information rapide et fiable de
l'attitude (angles dans l'espace) du système. Cette fusion peut être
gérée localement à l'aide d'une unité de traitement numérique intégrée
au capteur.

Ces capteurs se trouvent par exemple dans les \textbf{drones}, les
\textbf{smartphones} et les voitures disposant du contrôle de
trajectoire ESP. \\
\bottomrule
\end{longtable}

\hypertarget{autres-capteurs}{%
\subsection{Autres capteurs}\label{autres-capteurs}}

\hypertarget{capteurs-de-pression}{%
\subsubsection{Capteurs de pression}\label{capteurs-de-pression}}

Les capteurs de pressions sont de type tout ou rien~ou analogique. En
général ils fonctionnent par mesure de déformation d'un mécanisme
sensible à la pression (membrane équipée d'un ressort par exemple).

\begin{itemize}
\item
  \begin{quote}
  Les type TOR permettent de détecter un niveau choisi au préalable.
  \emph{Exemple~:} Détection de perte de pression d'huile dans un moteur
  automobile (voyant rouge en forme de burette d'huile).
  \end{quote}
\item
  \begin{quote}
  Les modèles analogiques permettent une mesure de la pression.
  \end{quote}
\end{itemize}

\hypertarget{capteur-de-duxe9bit}{%
\subsubsection{Capteur de débit}\label{capteur-de-duxe9bit}}

Les capteurs de débit permettent de mesure un débit de fluide, ils
peuvent utiliser un contact physique (hélice, pale) ou un système sans
contact (ultrason par exemple).

\hypertarget{capteurs-uxe0-ultrason-sonar-sg}{%
\subsubsection{Capteurs à ultrason / sonar
(SG)}\label{capteurs-uxe0-ultrason-sonar-sg}}

Les capteurs émettent un son et mesure son temps de
propagation/réflexion pour mesurer une distance à un objet, exemple~:
Aspirateur robot.

\hypertarget{ruxe9solveur-sg}{%
\subsubsection{Résolveur (SG)}\label{ruxe9solveur-sg}}

Ce sont des capteurs d'angles absolu par mesure du champ
électromagnétique par induction dans 2 bobines orientées à 90°.

\hypertarget{lidar}{%
\subsubsection{LIDAR}\label{lidar}}

Ces capteurs utilisent des lasers afin de mesurer des distances. En
disposant le laser sur une tourelle il est possible de cartographier un
environnement (scanner laser).

\hypertarget{traitement-de-linformation}{%
\section{Traitement de l'information}\label{traitement-de-linformation}}

Lorsque l'information est acquise dans un système, l'unité de traitement
réalise plusieurs opérations.

\begin{itemize}
\item
  Conversion analogique -- numérique~(CAN):

  \begin{itemize}
  \item
    échantillonnage~: cette opération consiste en un prélèvement de
    l'information à intervalle régulier.~;
  \item
    blocage~: pendant que le signal est converti, l'entrée est bloquée
    en l'état~;
  \item
    codage~: la valeur est codée en une information booléenne ou en
    information numérique. Suivant l'unité de traitement, le codage est
    limité à un certain nombre de bits, influant ainsi sur la valeur
    stockée.
  \end{itemize}
\item
  Stockage~: une fois numérisée l'information est stockée en mémoire
  (mémoire flash, RAM, disque-dur\ldots)
\item
  Traitement~: l'information peut alors être traitée à proprement parlé.
  Suivant les valeurs mesurées, l'unité de traitement pour alors
  modifier le comportement de la chaîne d'énergie.
\end{itemize}

Ces opérations vont être réalisées par un microcontrôleur ou un
microprocesseur. Ces composants sont programmables.

\begin{longtable}[]{@{}
  >{\raggedright\arraybackslash}p{(\columnwidth - 0\tabcolsep) * \real{1.00}}@{}}
\toprule
\endhead
\begin{minipage}[t]{\linewidth}\raggedright
\textbf{Exemples~de logiciels permettant de programmer des unités de
traitement}

\begin{longtable}[]{@{}ll@{}}
\toprule
\textbf{Programmation par un langage «~écrit~»} &
\includegraphics[width=0.79028in,height=0.95556in]{media/image60.png}\includegraphics[width=1.11628in,height=0.31425in]{media/image61.png}
\includegraphics[width=0.69767in,height=0.47471in]{media/image62.png} \\
\midrule
\endhead
\textbf{Programmation par langage graphique} &
\includegraphics[width=1.68835in,height=0.47674in]{media/image63.png}\includegraphics[width=1.0406in,height=0.27907in]{media/image64.png} \\
\textbf{Programmation par diagramme d'état, programmation par schéma
bloc} &
\includegraphics[width=1.69767in,height=0.51995in]{media/image65.png} \\
\textbf{Programmation par logigramme~: ISP Lever -- Lattice
Semiconductor} &
\includegraphics[width=1.73256in,height=0.59569in]{media/image66.png} \\
& \\
\bottomrule
\end{longtable}
\end{minipage} \\
\bottomrule
\end{longtable}

Un cours séparé couvre les principes et la modélisation des systèmes de
traitement de l'information.

\hypertarget{automate-programmables-industriels-api}{%
\subsection{Automate programmables industriels
(API)}\label{automate-programmables-industriels-api}}

Dans les systèmes industriels, le traitement de l'information est
réalisé par un automate programmable industriel. Il s'agit d'un système
électrique équipé d'entrées permettant de mesurer les états de capteurs
ou de détecteurs et de sorties permettant de piloter des modulateurs
d'énergie (relais électriques, distributeurs pneumatiques etc\ldots)

Le lien entre les entrées et les sorties se fait au moyen d'un programme
(programme graphique, codage\ldots.).

\begin{longtable}[]{@{}
  >{\raggedright\arraybackslash}p{(\columnwidth - 0\tabcolsep) * \real{1.00}}@{}}
\toprule
\endhead
\begin{minipage}[t]{\linewidth}\raggedright
\textbf{Exemples~de logiciels permettant de programmer des unités de
traitement}

Automate de la capsuleuse de bocaux

\begin{longtable}[]{@{}ll@{}}
\toprule
\includegraphics[width=3.50387in,height=1.35887in]{media/image67.png} &
\includegraphics[width=3.03488in,height=1.7789in]{media/image68.png} \\
\midrule
\endhead
\emph{\textbf{Automate TSX 17}} & \emph{\textbf{Représentation
schématique des entrées et sorties}} \\
& \\
\bottomrule
\end{longtable}
\end{minipage} \\
\bottomrule
\end{longtable}

\hypertarget{cartes-programmables-carte-arduino}{%
\subsection{Cartes programmables -- Carte
Arduino}\label{cartes-programmables-carte-arduino}}

Les caractéristiques de la carte Arduino UNO sont les suivantes~:

\begin{longtable}[]{@{}
  >{\raggedright\arraybackslash}p{(\columnwidth - 2\tabcolsep) * \real{0.50}}
  >{\raggedright\arraybackslash}p{(\columnwidth - 2\tabcolsep) * \real{0.50}}@{}}
\toprule
\endhead
\begin{minipage}[t]{\linewidth}\raggedright
\begin{itemize}
\item
  \begin{quote}
  Mémoire et microcontrôleur~:
  \end{quote}

  \begin{itemize}
  \item
    \begin{quote}
    Microcontrôleur ATmega328 cadencé à 16 MHz
    \end{quote}
  \item
    \begin{quote}
    Mémoire Flash 32ko (dont 0,5 ko pour le système d'amorçage)
    \end{quote}
  \item
    \begin{quote}
    SRAM~: 2ko
    \end{quote}
  \item
    \begin{quote}
    EEPROM 1ko
    \end{quote}
  \end{itemize}
\item
  \begin{quote}
  Entrées sorties numériques
  \end{quote}

  \begin{itemize}
  \item
    \begin{quote}
    14 dont 6 en MLI (PWM) indiquées \textasciitilde{} (40mA).
    \end{quote}
  \item
    \begin{quote}
    Ports Tx et Rx~: reprise du port série
    \end{quote}
  \end{itemize}
\end{itemize}
\end{minipage} & \begin{minipage}[t]{\linewidth}\raggedright
\begin{itemize}
\item
  \begin{quote}
  Alimentation~:
  \end{quote}

  \begin{itemize}
  \item
    \begin{quote}
    Alimentation par le port USB~: 5V, 500mA
    \end{quote}
  \item
    \begin{quote}
    Alimentation externe en 7 à 12 V (2,1 mm)
    \end{quote}
  \item
    \begin{quote}
    Reprise de l'alimentation externe
    \end{quote}
  \item
    \begin{quote}
    Alimentation externe régulée en 5V/500mA et 3,3 V/50mA.
    \end{quote}
  \end{itemize}
\item
  \begin{quote}
  Entrées analogiques~:
  \end{quote}

  \begin{itemize}
  \item
    \begin{quote}
    5 entrées analogiques
    \end{quote}
  \end{itemize}
\end{itemize}
\end{minipage} \\
\bottomrule
\end{longtable}

\includegraphics[width=4.89535in,height=2.8114in]{media/image69.png}

\hypertarget{traitement-avec-cuxe2blage-uxe9lectrique-pneumatique-et-hydraulique}{%
\subsubsection{Traitement avec câblage électrique, pneumatique et
hydraulique}\label{traitement-avec-cuxe2blage-uxe9lectrique-pneumatique-et-hydraulique}}

En utilisant des pré actionneurs électriques, pneumatiques ou
hydrauliques il est possible de réaliser un traitement des informations
en «~logique câblée~». Ce type de traitement est couteux et fastidieux
et on le réserve en général pour des prises de décision simple liées à
la sécurité \emph{Par exemple déclenchement d'une interruption de
production en cas de détection d'incendie.}

\hypertarget{traitement-avec-un-automate-logique}{%
\subsubsection{Traitement avec un automate
logique}\label{traitement-avec-un-automate-logique}}

Plus simple que les automates programmables industriels ces automates ne
disposent pas de mémoire. Ils peuvent servir pour une installation
simple et peu couteuse ou traiter uniquement un aspect de sécurité de
l'installation par exemple.

\hypertarget{traitement-avec-un-fpga}{%
\subsubsection{Traitement avec un FPGA}\label{traitement-avec-un-fpga}}

Les FPGA sont des puces électroniques dont l'aspect est proche des
microcontrôleurs. A la différence de ceux-ci leur circuit interne sont
modifiables par une interface PC. Les connexions entre les transistors
peuvent être reconfigurées afin d'obtenir le traitement souhaité. Cela
permet de développer des traitements de l'information performants
adaptés à une application particulière, ou un logiciel embarqué exécuté
sur microcontrôleur ne se montrerait pas suffisamment réactif. Les FPGA
sont particulièrement adapté aux petites séries ne permettant pas une
mise en production de puce spécifique (ASIC). \emph{Exemple~: Guidage de
fusée}

\hypertarget{traitement-avec-un-digital-signal-processor-dsp}{%
\subsubsection{Traitement avec un Digital Signal Processor
DSP}\label{traitement-avec-un-digital-signal-processor-dsp}}

Pour traiter un flux d'information massif en temps réel comme un signal
audio un microcontrôleur n'est pas adapté. Le volume d'information à
traiter en peu de temps est trop important. Des puces DSP dédiées sont
alors utilisées.

\hypertarget{fonction-convertir}{%
\section{Fonction Convertir}\label{fonction-convertir}}

Tous les systèmes physiques ont au moins besoin d'un actionneur pour
agir sur la matière d'œuvre, cet actionneur ne crée pas d'énergie du
néant, il convertit une source d'énergie primaire en énergie adaptée à
l'action souhaitée. Dans le laboratoire les actionneurs seront souvent
des moteurs électriques car cette énergie est directement disponible.

\hypertarget{les-moteurs-uxe9lectriques}{%
\subsection{Les moteurs électriques}\label{les-moteurs-uxe9lectriques}}

\begin{longtable}[]{@{}
  >{\raggedright\arraybackslash}p{(\columnwidth - 2\tabcolsep) * \real{0.50}}
  >{\raggedright\arraybackslash}p{(\columnwidth - 2\tabcolsep) * \real{0.50}}@{}}
\toprule
\endhead
\includegraphics[width=2.91947in,height=2.25279in]{media/image70.jpeg} &
Le moteur électrique est constitué d'aimants et de fils enroulés. Il se
base sur la force de Laplace : tout conducteur parcouru par un courant
et plongé dans un champ magnétique reçoit une force, la force de
Laplace, proportionnelle à l'intensité du courant et du champ
magnétique.

Un système particulier permet de faire varier tourner le champ
magnétique afin de générer une force de Laplace motrice pour le
mouvement de rotation.

Les moteurs électriques sont des machines réversibles, il est possible
de les faire tourner avec un dispositif mécanique pour récupérer une
énergie électrique. On parle de fonctionnement en génératrice. \\
\bottomrule
\end{longtable}

\hypertarget{le-moteur-uxe0-courant-continu}{%
\subsubsection{Le moteur à courant
continu}\label{le-moteur-uxe0-courant-continu}}

Les aimants sont liés au stator (partie fixe) et les bobines qui génère
le champ tournant sont liées au rotor. Un dispositif mécanique, les
balais, permettent de faire tourner le champ électrique pour qu'il soit
toujours moteur. Une source de tension continue est suffisante et ce
moteur dispose donc de 2 fil pour l'apport en énergie. Des capteurs
optionnels peuvent augmenter le nombre de connexions.

La modélisation de cet actionneur est détaillée dans un cours. On
observe une proportionnalité du courant avec le couple délivré
(constante de couple Ki) et une proportionnalité de la tension de~force
contre électro motrice~à la vitesse de rotation (constante Ke).
Lorsqu'elles sont exprimées dans un même système d'unité ces constantes
sont proches ou égales.

\hypertarget{le-moteur-brushless-sans-broches}{%
\subsubsection{Le moteur brushless (sans
broches)}\label{le-moteur-brushless-sans-broches}}

\begin{longtable}[]{@{}
  >{\raggedright\arraybackslash}p{(\columnwidth - 2\tabcolsep) * \real{0.50}}
  >{\raggedright\arraybackslash}p{(\columnwidth - 2\tabcolsep) * \real{0.50}}@{}}
\toprule
\endhead
\includegraphics[width=3.79554in,height=2.84745in]{media/image71.jpeg} &
Ce moteur ne dispose pas de dispositif mécanique permettant de faire
tourner le champ. Les aimants sont liés au rotor et 3 connexions
permettent d'alimenter 3 paires de poles (bobines) distinctes. En
alimentant la bonne paire de pole par rapport à la position du rotor
(aimant) le moteur génère un couple positif (il fonctionne).

Afin de choisir la bonne paire de pole il est nécessaire de connaitre la
position du rotor. Un capteur ou des techniques mesurant les courant
d'induction dans les phases permettent de connaitre cette position afin
de piloter le moteur.

\emph{Voir variateur, section «~moduler~»} \\
\bottomrule
\end{longtable}

\hypertarget{le-moteur-asynchrone}{%
\subsubsection{Le moteur asynchrone}\label{le-moteur-asynchrone}}

Ce type de moteur utilise un champ magnétique dont la vitesse de
rotation n'est pas synchronisée avec la vitesse de rotation. On parle de
glissement. Cette machine est souvent privilégiée notamment pour les
fortes puissances (ferroviaire).

Comme le moteur brushless une électronique de puissance sophistiquée est
nécessaire pour un pilotage performant. Une connexion à un courant
alternatif permettra cependant de la faire tourner sans pilotage.

\hypertarget{le-moteur-pas-uxe0-pas}{%
\subsubsection{Le moteur pas à pas}\label{le-moteur-pas-uxe0-pas}}

\begin{longtable}[]{@{}lll@{}}
\toprule
\endhead
\includegraphics[width=2.0855in,height=2.0855in]{media/image72.png} &
\includegraphics[width=1.875in,height=1.90625in]{media/image73.png} & Ce
moteur est proche dans son architecture du moteur brushless. Il y a en
plus de poles. Les performances (puissance massique, puissance totale,
efficacité énergétique) sont plus faibles mais le couple est important
ce qui permet de ne pas utiliser de réducteur pour de nombreuses
applications. Aussi chaque pas représente un petit incrément angulaire
ce qui est intéressant pour des activités de précision (pilotage d'un
téléscope de hobbysite, imprimante 3D). \\
\bottomrule
\end{longtable}

\hypertarget{actionneurs-linuxe9aires}{%
\subsubsection{Actionneurs linéaires}\label{actionneurs-linuxe9aires}}

Il est possible de faire translater un aimant en le soumettant à un
champ magnétique afin de créer un actionneur linéaire. Ces actionneurs
sont rapides mais l'amplitude possible et la force générées sont en
général limitée. \emph{Exemple~:} prototype d'actionneur de soupapes
pour moteurs sans distribution, prototype de suspensions intelligentes.

La plupart des vérins électrique utilisent en revanche un moteur
électrique tournant associé à un ensemble vis-écrou

\hypertarget{convertisseurs-duxe9nergie-pneumatique-ou-hydraulique}{%
\subsection{Convertisseurs d'énergie pneumatique ou
hydraulique}\label{convertisseurs-duxe9nergie-pneumatique-ou-hydraulique}}

\begin{longtable}[]{@{}
  >{\raggedright\arraybackslash}p{(\columnwidth - 2\tabcolsep) * \real{0.50}}
  >{\raggedright\arraybackslash}p{(\columnwidth - 2\tabcolsep) * \real{0.50}}@{}}
\toprule
\endhead
Un vérin est un actionneur utilisant de l'énergie pneumatique ou
hydraulique pour produire une énergie mécanique lors d'un déplacement
linéaire ou rotatif limité à sa course. Le vérin permet de convertir de
l'énergie pneumatique (ou hydraulique) en énergie mécanique.

\includegraphics[width=2.9994in,height=0.52326in]{media/image74.png}

Dans les deux cas le produit des deux valeurs donne une puissance, la
puissance \(\mathbf{P \cdot Q}\) pneumatique étant convertie en
puissance \(\mathbf{F \cdot V}\) mécanique. Il est à noter que le
rendement de ces actionneurs est mauvais (\(\eta = 0,5\) environ) : une
grande partie de l'énergie est perdue sous forme d'énergie calorifique
et lors de la mise à l'échappement de l'air comprimé. En prenant en
compte le rendement du compresseur (\(\eta = 0,4\)), on obtient un
rendement global très faible pour la chaîne d'action pneumatique
(\(\eta = 0,2\)). &
\includegraphics[width=1.60465in,height=1.07872in]{media/image75.png}

\includegraphics[width=2.54651in,height=1.82389in]{media/image76.png} \\
\bottomrule
\end{longtable}

\begin{longtable}[]{@{}lll@{}}
\toprule
\includegraphics[width=2.18753in,height=0.78785in]{media/image77.png} &
\includegraphics[width=1.94972in,height=0.85305in]{media/image78.png} &
\includegraphics[width=1.95945in,height=0.94776in]{media/image79.png} \\
\midrule
\endhead
\includegraphics[width=0.7365in,height=0.38835in]{media/image80.png} &
\includegraphics[width=0.68657in,height=0.4208in]{media/image81.png} &
\includegraphics[width=0.83815in,height=0.50289in]{media/image82.png} \\
\emph{\textbf{Vérin linéaire simple effet}} & \emph{\textbf{Vérin
linéaire double effet}} & \emph{\textbf{Vérin rotatif double effet}} \\
\bottomrule
\end{longtable}

\textbf{La force} délivrée par un vérin est donnée par \(F = PS\) avec
\(F\) la force en newtons, \(S\) la section du vérin en m² et \(P\) la
pression en Pascals.

\textbf{Le débit volumique} est donné par \(Q = VS\) avec \(Q\) en
m.\textsuperscript{3}s\textsuperscript{-1} et \(V\) la vitesse de
déplacement du vérin en m.s\textsuperscript{-1}.

\textbf{La cylindrée} est donnée par \(C = S\ c\) avec \(S\) en m² et
\emph{c}, la course en m.

\begin{longtable}[]{@{}
  >{\raggedright\arraybackslash}p{(\columnwidth - 2\tabcolsep) * \real{0.50}}
  >{\raggedright\arraybackslash}p{(\columnwidth - 2\tabcolsep) * \real{0.50}}@{}}
\toprule
\endhead
\includegraphics[width=1.42708in,height=0.71875in]{media/image83.png} &
On note D le diamètre du vérin et d le diamètre de la tige.

Dans la chambre de gauche l'effort est donné par
\(F_{g} = p_{g}\frac{\pi D^{2}}{4}\).

Dans la chambre de droite l'effort est donné par
\(F_{d} = p_{g}\frac{\pi\left( D^{2} - d^{2} \right)}{4}\). \\
\bottomrule
\end{longtable}

\hypertarget{moteurs-thermiques}{%
\subsection{Moteurs thermiques}\label{moteurs-thermiques}}

\begin{longtable}[]{@{}ll@{}}
\toprule
\endhead
\includegraphics[width=2.79532in,height=3.1643in]{media/image84.jpeg} &
Les moteurs thermiques permettent de récupérer une énergie mécanique, en
général en rotation, à partir d'une source d'énergie chimique. La
combustion peut être externe comme dans une machine à vapeur. Dans un
moteur traditionnel (auto bateau\ldots) ou dans une turbine, un
réacteur, la combustion est interne. Un dispositif mécanique (système
bielle manivelle, ailettes) permet la récupération de l'énergie
mécanique emmagasiné sous forme de pression lors de la combustion. Les
combustions internes sont généralement préférées car elles diminuent les
pertes par échange thermique. Pour aller plus loin~: moteur 4 temps/
moteur 2 temps, cycle Diesel, cycle Beau de Rochas/Otto, moteur rotatif,
distribution. \\
\bottomrule
\end{longtable}

\hypertarget{fonction-transmettre}{%
\section{Fonction Transmettre}\label{fonction-transmettre}}

\hypertarget{eluxe9ments-de-liaisons}{%
\subsection{Eléments de liaisons}\label{eluxe9ments-de-liaisons}}

Transmission d'une énergie en rotation.

\includegraphics[width=6.88958in,height=2.02437in]{media/image85.png}

\hypertarget{transmission-par-engrenages}{%
\subsection{Transmission par
engrenages}\label{transmission-par-engrenages}}

\hypertarget{engrenage-uxe0-contact-extuxe9rieur}{%
\subsubsection{Engrenage à contact
extérieur}\label{engrenage-uxe0-contact-extuxe9rieur}}

En considérant deux roues dentées \(a\) et \(b\) de nombre de dents
respectifs \(Z_{a}\) et \(Z_{b}\), on trouve~:

\begin{longtable}[]{@{}
  >{\raggedright\arraybackslash}p{(\columnwidth - 2\tabcolsep) * \real{0.50}}
  >{\raggedright\arraybackslash}p{(\columnwidth - 2\tabcolsep) * \real{0.50}}@{}}
\toprule
\endhead
& \begin{minipage}[t]{\linewidth}\raggedright
\begin{quote}
La relations des vitesses~:
\end{quote}

\[\boxed{\frac{\omega_{a}}{\omega_{b}} = - \frac{Z_{b}}{Z_{a}}}\]

\begin{quote}
La relation sur les couples~\(
\)(frottements négligés) :
\end{quote}

\[\boxed{\frac{C_{a}}{C_{b}} = - \frac{Z_{a}}{Z_{b}}}\]
\end{minipage} \\
\bottomrule
\end{longtable}

\hypertarget{engrenage-uxe0-contact-intuxe9rieur}{%
\subsubsection{Engrenage à contact
intérieur}\label{engrenage-uxe0-contact-intuxe9rieur}}

En considérant deux roues dentées \(a\) et \(b\) de nombre de dents
respectifs \(Z_{a}\) et \(Z_{b}\), on trouve~:

\begin{longtable}[]{@{}
  >{\raggedright\arraybackslash}p{(\columnwidth - 2\tabcolsep) * \real{0.50}}
  >{\raggedright\arraybackslash}p{(\columnwidth - 2\tabcolsep) * \real{0.50}}@{}}
\toprule
\endhead
& \begin{minipage}[t]{\linewidth}\raggedright
\begin{quote}
La relations des vitesses~:
\end{quote}

\[\boxed{\frac{\omega_{a}}{\omega_{b}} = \frac{Z_{b}}{Z_{a}}}\]

\begin{quote}
La relation sur les couples~\(
\)(frottements négligés) :
\end{quote}

\[\boxed{\frac{C_{a}}{C_{b}} = \frac{Z_{a}}{Z_{b}}}\]
\end{minipage} \\
\bottomrule
\end{longtable}

\hypertarget{roues-et-vis-sans-fin}{%
\subsubsection{Roues et vis sans fin}\label{roues-et-vis-sans-fin}}

\includegraphics[width=6.88958in,height=1.65739in]{media/image89.png}

\hypertarget{renvoi-dangle}{%
\subsubsection{Renvoi d'angle~:}\label{renvoi-dangle}}

\includegraphics[width=6.88958in,height=1.62394in]{media/image90.png}

\hypertarget{transmission-par-courroies-et-chaines}{%
\subsection{Transmission par courroies et
chaines}\label{transmission-par-courroies-et-chaines}}

\includegraphics[width=2.38958in,height=1.22917in]{media/image92.jpeg}Le
nombre de dents étant proportionnel au rayon, on trouve~:

\begin{longtable}[]{@{}
  >{\raggedright\arraybackslash}p{(\columnwidth - 2\tabcolsep) * \real{0.50}}
  >{\raggedright\arraybackslash}p{(\columnwidth - 2\tabcolsep) * \real{0.50}}@{}}
\toprule
\endhead
& \begin{minipage}[t]{\linewidth}\raggedright
\begin{quote}
La relations des vitesses~:
\end{quote}

\[\boxed{\frac{\omega_{a}}{\omega_{b}} = \frac{Z_{b}}{Z_{a}} = \frac{R_{b}}{R_{a}}}\]

\begin{quote}
La relation sur les couples~\(
\)(frottements négligés) :
\end{quote}

\[\boxed{\frac{C_{a}}{C_{b}} = \frac{Z_{a}}{Z_{b}} = \frac{R_{a}}{R_{b}}}\]
\end{minipage} \\
\bottomrule
\end{longtable}

\hypertarget{boites-de-vitesse}{%
\subsection{Boites de vitesse}\label{boites-de-vitesse}}

\begin{longtable}[]{@{}ll@{}}
\toprule
\endhead
\includegraphics[width=2.83342in,height=1.75in]{media/image93.jpeg} & La
boite de vitesse permet d'obtenir plusieurs rapports de réduction
distinct ce qui est pratique lorsque les vitesses de sortie et de
l'actionneur/convertisseur sont susceptible de varier avec beaucoup
d'amplitude. \\
\bottomrule
\end{longtable}

\hypertarget{convertisseurs-multiples}{%
\subsection{Convertisseurs multiples}\label{convertisseurs-multiples}}

Des techniques utilisant plusieurs conversions permettent d'obtenir des
transmissions sophistiquées. Dans certains contextes il faudra faire
attention à ne pas trop dégrader l'efficacité car chaque conversion
d'énergie ajoute des pertes.

\begin{longtable}[]{@{}ll@{}}
\toprule
\endhead
\includegraphics[width=2.60486in,height=1.94792in]{media/image94.jpeg} &
L'association d'une pompe réglable et d'un moteur hydraulique permettent
d'obtenir une transmission continument variable dans la transmission
pour moto Honda HFT. \\
\bottomrule
\end{longtable}

Dans les locomotives diesel et les navires utilisant des PODs de
propulsion une génératrice électrique est associé au moteur thermique
afin d'alimenter le moteur électrique de propulsion. On parle aussi
«~d'hybride série~». Pour les navire cela permet de supprimer la ligne
d'arbre qui est nuisible pour le rendement hydrodynamique. Poru les
locomotive cela permet d'obtenir le fonctionnement souple important pour
le transport de passer et la transmission sur rail.

\hypertarget{fonction-modulercommuter}{%
\section{Fonction moduler/commuter}\label{fonction-modulercommuter}}

\hypertarget{introduction}{%
\subsection{Introduction}\label{introduction}}

Dans la chaîne fonctionnelle, le modulateur d'énergie (ou distributeur
d'énergie ou pré actionneurs) est le composant qui fait le lien entre la
chaîne d'information et la chaîne d'énergie. Ainsi, à partir d'une
faible puissance énergétique provenant de la fonction «~Traiter~» (l'API
ou la carte de commande), il peut faire transiter une grande puissance
(provenant de la fonction «~Alimenter~» ou «~Stocker~».

\begin{longtable}[]{@{}
  >{\raggedright\arraybackslash}p{(\columnwidth - 0\tabcolsep) * \real{1.00}}@{}}
\toprule
\endhead
\textbf{Définition : Tout ou rien -- Variateur}

Les distributeurs «~tout ou rien~» permettent d'envoyer toute l'énergie
de l'alimentation vers le convertisseur.

Les distributeurs de type «~variateur~» permettent de moduler l'énergie
envoyée au convertisseur. \\
\bottomrule
\end{longtable}

\begin{longtable}[]{@{}
  >{\raggedright\arraybackslash}p{(\columnwidth - 0\tabcolsep) * \real{1.00}}@{}}
\toprule
\endhead
\textbf{Exemples :}

Un interrupteur de lumière peut être considéré comme un distributeur
tout ou rien.

Le variateur d'une lampe halogène peut être considéré comme un \ldots{}
variateur. \\
\bottomrule
\end{longtable}

\begin{longtable}[]{@{}
  >{\raggedright\arraybackslash}p{(\columnwidth - 0\tabcolsep) * \real{1.00}}@{}}
\toprule
\endhead
\textbf{Définition : Monostable -- Bistable}

Un pré-actionneur est dit monostable s'il a besoin d'un ordre pour le
faire passer de sa position de repos à sa position de travail et que le
retour à sa position de repos s'effectue automatiquement lorsque l'ordre
disparait : \textbf{il n'est stable que dans une seule position}.

Un pré-actionneur est dit bistable s'il a besoin d'un ordre pour passer
de sa position repos à sa position travail et qu'il reste en position
travail à la disparition de cet ordre. Il ne peut revenir à sa position
repos que s'il reçoit un second ordre : \textbf{il est stable dans les
deux positions}. \\
\bottomrule
\end{longtable}

\begin{longtable}[]{@{}
  >{\raggedright\arraybackslash}p{(\columnwidth - 0\tabcolsep) * \real{1.00}}@{}}
\toprule
\endhead
\textbf{Exemple~:}

Un interrupteur de lumière peut être considéré comme un distributeur
bistable. Il faut appuyer dessus pour allumer une lumière et appuyer une
seconde fois pour l'éteindre. \\
\bottomrule
\end{longtable}

\hypertarget{les-modulateurs-uxe9lectriques}{%
\subsection{Les modulateurs
électriques}\label{les-modulateurs-uxe9lectriques}}

\hypertarget{le-relai-ou-contacteur-de-puissance}{%
\subsubsection{Le relai (ou contacteur de
puissance)}\label{le-relai-ou-contacteur-de-puissance}}

Le relai est un dispositif électrique permettant de commander un circuit
de commande ou un circuit de puissance.

Le circuit secondaire alimente la partie que l'on veut commander.
Lorsque la bobine est alimentée le levier pivote provoquant la fermeture
du contact. Certains relais peuvent aussi être actionnés manuellement.

\begin{longtable}[]{@{}lll@{}}
\toprule
\endhead
\includegraphics[width=1.21388in,height=1.59525in]{media/image95.png} &
\includegraphics[width=2.34736in,height=1.69567in]{media/image96.png} &
\includegraphics[width=2.4015in,height=1.7093in]{media/image97.png} \\
\bottomrule
\end{longtable}

\begin{longtable}[]{@{}
  >{\raggedright\arraybackslash}p{(\columnwidth - 4\tabcolsep) * \real{0.33}}
  >{\raggedright\arraybackslash}p{(\columnwidth - 4\tabcolsep) * \real{0.33}}
  >{\raggedright\arraybackslash}p{(\columnwidth - 4\tabcolsep) * \real{0.33}}@{}}
\toprule
\begin{minipage}[b]{\linewidth}\raggedright
\begin{quote}
\textbf{Contacteur électrique monostable}
\end{quote}
\end{minipage} & & Quand la bobine reçoit un ordre de marche (appui sur
le bouton poussoir) la bobine est alimentée par un courant, créant ainsi
un champ magnétique. Le champ magnétique créé dans la bobine provoque le
déplacement du noyau de fer doux vers le haut. Le contact de puissance
est alors fermé.

Le moteur est alimenté puis mis en rotation.

Quand l'ordre de marche est interrompu (bouton relâché), le circuit de
commande est ouvert. La bobine n'est plus alimentée et le ressort de
rappel fait redescendre le noyau de fer doux.

Le circuit de puissance s'ouvre et le moteur n'est plus alimenté.

Ce contacteur est monostable car il alimente en énergie électrique le
moteur tant que l'ordre est maintenu. \\
\midrule
\endhead
\begin{minipage}[t]{\linewidth}\raggedright
\begin{quote}
\textbf{Contacteur électrique bistable}
\end{quote}
\end{minipage} &
\includegraphics[width=3.07395in,height=3.0597in]{media/image99.jpeg} &
Ce contacteur est bistable : il faut un ordre (court) pour que le moteur
soit alimenté. Le moteur continue à être alimenté même quand l'ordre de
marche a disparu. Il faut un ordre d'arrêt (court) pour que le moteur ne
soit plus alimenté. \\
\bottomrule
\end{longtable}

\begin{longtable}[]{@{}
  >{\raggedright\arraybackslash}p{(\columnwidth - 0\tabcolsep) * \real{1.00}}@{}}
\toprule
\endhead
\begin{minipage}[t]{\linewidth}\raggedright
\textbf{Symbolisation des contacts}

\begin{longtable}[]{@{}lll@{}}
\toprule
\endhead
\includegraphics[width=0.79201in,height=0.7874in]{media/image100.png} &
\includegraphics[width=0.76541in,height=0.7874in]{media/image101.png} &
\includegraphics[width=1.36856in,height=0.98425in]{media/image102.png} \\
\bottomrule
\end{longtable}
\end{minipage} \\
\bottomrule
\end{longtable}

\hypertarget{le-hacheur-convertisseur-statique}{%
\subsubsection{Le hacheur (convertisseur
statique)}\label{le-hacheur-convertisseur-statique}}

\begin{longtable}[]{@{}
  >{\raggedright\arraybackslash}p{(\columnwidth - 2\tabcolsep) * \real{0.50}}
  >{\raggedright\arraybackslash}p{(\columnwidth - 2\tabcolsep) * \real{0.50}}@{}}
\toprule
\endhead
\includegraphics[width=2.0486in,height=1.82318in]{media/image103.png} &
Lorsqu'on souhaite contrôler la fréquence de rotation d'un moteur à
courant continu ou moduler la puissance électrique s'appliquant sur une
charge, il est nécessaire de moduler sa tension d'alimentation. On
pourrait pour cela utiliser un pont diviseur, mais cette technologie
serait très énergivore à cause des pertes joules qui apparaitraient dans
les résistances. Historiquement des transistors linéaires étaient
utilisés mais ils sont couteux et peu fiables. On utilise désormais un
hacheur.

Un hacheur est composé de transistors «~tout ou rien~» utilisant la
technologie «~MOSFET~». Cette technologie permet de commuter (laisser
passer ou non) des courants importants avec une bonne fiabilité, un bon
rendement et une rapidité de commutation bien supérieure au relai. Une
bonne coordination de l'ouverture et de la fermeture de ces
interrupteurs permet de générer une tension ayant une forme de créneau
où les temps à l'état bas et à l'état haut sont réglables. \\
\bottomrule
\end{longtable}

Le hacheur est caractérisé par sa période de hachage (980 Hz pour une
carte Arduino Leonardo), ainsi que par le rapport cyclique (variable),
définit par le pourcentage de la période passé à l'état haut. Il envoie
ainsi un signal appelé MLI (Modulation de Largeur d'Impulsion) ou PWM
(Pulse Width Modulation).

\begin{longtable}[]{@{}lll@{}}
\toprule
\includegraphics[width=2.13954in,height=1.17904in]{media/image104.png} &
\includegraphics[width=2.17793in,height=1.56693in]{media/image105.png} &
\includegraphics[width=2.82292in,height=1.93958in]{media/image106.png} \\
\midrule
\endhead
\emph{Modèle simplifié du pilotage d'un moteur électrique à courant
continu} & \emph{Schéma proche du câblage réel. L'interrupteur K est
commandé par le signal MLI} & \emph{Signal MLI avec 3 rapport cycliques
distincts} \\
\bottomrule
\end{longtable}

Dans le cas précédent, si le moteur est alimenté par un créneau valant
24 V 25\% du temps. Il est donc alimenté en 6 V en moyenne.

\hypertarget{londuleur-variateur}{%
\subsubsection{L'onduleur (variateur)}\label{londuleur-variateur}}

Les moteurs triphasés sont physiquement alimentés par 3 fils. La tension
est sinusoïdale et décalée dans chacun d'entre eux d'un tiers de
période. Afin de générer un signal sinusoïdal de fréquence et
d'amplitude voulue on a recours à un onduleur.

Pour cela, en règle générale, on redresse la tension issue de
l'alimentation du secteur puis on régénère un signal avec l'onduleur.

\begin{longtable}[]{@{}ll@{}}
\toprule
\endhead
\includegraphics[width=3.95349in,height=0.98369in]{media/image107.png} &
\includegraphics[width=2.39535in,height=0.89582in]{media/image108.png} \\
\bottomrule
\end{longtable}

Le variateur est une forme d'onduleur qui permet de piloter avec
précision la vitesse ou la position d'un moteur triphasé. Pour cela il
utilise en général un capteur afin de connaitre la position du rotor et
alimenter la bonne phase. Certaines technologies peuvent déterminer la
position du rotor sans capteur en mesurant les effets d'induction dans
les phases (montée et descente de courant lorsque l'on commute la phase)
\emph{technologie sensorless}.

\hypertarget{notion-de-schuxe9ma-uxe9lectrique}{%
\subsubsection{Notion de schéma
électrique}\label{notion-de-schuxe9ma-uxe9lectrique}}

\begin{longtable}[]{@{}ll@{}}
\toprule
\textbf{Inversion de sens d'un moteur CC.} & \textbf{Inversion de sens
d'un moteur triphasé asynchrone} \\
\midrule
\endhead
\includegraphics[width=2.98248in,height=1.96512in]{media/image109.png} &
\includegraphics[width=3.75718in,height=2.13903in]{media/image110.png} \\
\bottomrule
\end{longtable}

\hypertarget{les-modulateurs-pneumatiques-et-hydrauliques}{%
\subsection{Les modulateurs pneumatiques et
hydrauliques}\label{les-modulateurs-pneumatiques-et-hydrauliques}}

\begin{longtable}[]{@{}
  >{\raggedright\arraybackslash}p{(\columnwidth - 0\tabcolsep) * \real{1.00}}@{}}
\toprule
\endhead
\textbf{Définition : Énergie hydraulique et pneumatique}

\textbf{Énergie pneumatique} : le fluide utilisé est de l'air comprimé.

\textbf{Énergie hydraulique} : le fluide utilisé est une huile
hydraulique minérale ou difficilement inflammable (aqueuse ou non). \\
\bottomrule
\end{longtable}

\hypertarget{les-distributeurs}{%
\subsection{Les distributeurs}\label{les-distributeurs}}

\begin{longtable}[]{@{}
  >{\raggedright\arraybackslash}p{(\columnwidth - 2\tabcolsep) * \real{0.50}}
  >{\raggedright\arraybackslash}p{(\columnwidth - 2\tabcolsep) * \real{0.50}}@{}}
\toprule
Les distributeurs sont les préactionneurs des vérins pneumatiques et
hydrauliques.

Ils servent d'«~aiguillages~» en dirigeant le fluide dans certaines
directions. Les plus utilisés sont les distributeurs à tiroir. &
\includegraphics[width=1.81782in,height=1.36364in]{media/image111.png} \\
\midrule
\endhead
\includegraphics[width=3.14961in,height=1.88001in]{media/image112.png}

\emph{Vérin simple effet et distributeur 3/2 monostable NF à commande
manuelle par bouton} &
\includegraphics[width=3.14961in,height=1.85091in]{media/image113.png}

\emph{Vérin double effet et distributeur 5/2 monostable à commande
manuelle par bouton} \\
\includegraphics[width=5.90551in,height=3.9183in]{media/image114.png}

\emph{Vérin double effet à amortissement réglable et distributeur 5/2
bistable à commande électropneumatique} & \\
\bottomrule
\end{longtable}

\hypertarget{duxe9signation-des-distributeurs}{%
\subsection{Désignation des
distributeurs}\label{duxe9signation-des-distributeurs}}

Lors de l'élaboration des schémas, il n'est pas possible de représenter
le distributeur, ainsi que les autres composants, sous leurs formes
commerciales. De ce fait, l'utilisation de symboles normalisés simplifie
la lecture et la compréhension des systèmes. Cette représentation
utilise la symbolisation par cases.

Un distributeur se représente sur les côtés droit et/ou gauche (comme
dans la réalité) par des pilotages. Ils permettent au tiroir de se
déplacer afin de mettre en communication les différents orifices.

\begin{longtable}[]{@{}
  >{\raggedright\arraybackslash}p{(\columnwidth - 0\tabcolsep) * \real{1.00}}@{}}
\toprule
\endhead
\textbf{Désignation}

La désignation d'un distributeur permet de mettre en évidence le nombre
d'orifices du distributeur, le nombre de positions, le type de commande
et son état (monostable ou bistable). \\
\bottomrule
\end{longtable}

\includegraphics[width=5.54274in,height=5.96101in]{media/image115.png}

\includegraphics[width=6.32735in,height=3.84416in]{media/image116.png}

\hypertarget{fonction-stocker}{%
\section{Fonction stocker}\label{fonction-stocker}}

\hypertarget{piles-et-batteries}{%
\subsection{Piles et batteries}\label{piles-et-batteries}}

\begin{longtable}[]{@{}ll@{}}
\toprule
\endhead
\includegraphics[width=4.375in,height=1.64113in]{media/image117.png} &
Stockage d'énergie électrique par des techniques chimique. En première
approximation ces éléments se comportent comme des sources de tension
parfaite. En revanche lors d'une utilisation trop longue par rapport à
la capacité ou trop intense par rapport au courant maximum prévu, une
chute plus ou moins importante de tension peut se produire. \\
\bottomrule
\end{longtable}

\hypertarget{energies-pneumatiques-et-hydrauliques}{%
\subsection{Energies pneumatiques et
hydrauliques}\label{energies-pneumatiques-et-hydrauliques}}

Les énergies pneumatiques et hydrauliques sont obtenues grâce à des
compresseurs (ou des pompes) actionnés par un moteur électrique ou
thermique. Dans les systèmes pneumatiques, la circulation d'air se fait
généralement en circuit ouvert. Dans le cas des systèmes hydrauliques,
le fluide est en circuit fermé. Cela impose des conditions sur les
constituants des réseaux. Ces motopompes peuvent constituer un nouveau
bloc «~alimentation~» en amont de la chaine d'énergie.

\begin{longtable}[]{@{}lll@{}}
\toprule
\includegraphics[width=1.16072in,height=0.74419in]{media/image118.png} &
\includegraphics[width=0.88343in,height=0.94186in]{media/image120.png} &
\includegraphics[width=1.04671in,height=0.72385in]{media/image121.png} \\
\midrule
\endhead
\emph{Symbole d'un moteur} & \emph{Symbole d'une pompe à deux sens de
rotation et deux sens de flux} & \emph{Symbole d'un groupe moteur +
pompe} \\
\bottomrule
\end{longtable}

L'air est en général stocké dans une citerne sous pression qui fait
office de tampon entre les systèmes et la pompe. L'huile est en général
stockée à pression atmosphérique dans des réservoirs (parfois appelé
«~bâche~»). Le tampon entre la pompe et le système est un
«~accumulateur~hydraulique»~: c'est un réservoir qui contient un gaz
sous pression d'un coté, l'huile que l'on souhaite conserver sous
pression de l'autre et une membrane souple entre les deux.

\begin{longtable}[]{@{}llll@{}}
\toprule
\includegraphics[width=1.23955in,height=1.30233in]{media/image122.png} &
\includegraphics[width=1.4186in,height=1.24551in]{media/image123.png} &
\includegraphics[width=2.47917in,height=2.20833in]{media/image124.png} &
\includegraphics[width=1.03488in,height=0.22687in]{media/image125.png} \\
\midrule
\endhead
& & & \emph{Symbole réservoir} \\
& & &
\includegraphics[width=0.45652in,height=1.10368in]{media/image126.png} \\
\emph{Réservoir de 50 à 25 000 L} & \emph{Compresseur 100 L -- 10 bars}
& \emph{Accumulateur hydraulique} & \emph{Symbole accumulateur} \\
\bottomrule
\end{longtable}

\hypertarget{stockage-par-gravituxe9}{%
\subsection{Stockage par gravité}\label{stockage-par-gravituxe9}}

\begin{longtable}[]{@{}
  >{\raggedright\arraybackslash}p{(\columnwidth - 2\tabcolsep) * \real{0.50}}
  >{\raggedright\arraybackslash}p{(\columnwidth - 2\tabcolsep) * \real{0.50}}@{}}
\toprule
\endhead
\begin{minipage}[t]{\linewidth}\raggedright
\includegraphics[width=4.22581in,height=1.43132in]{media/image127.jpeg}
En plaçant une masse en hauteur on peut récupérer son énergie
potentielle de pesanteur.

\begin{itemize}
\item
  Un barrage hydroélectrique ou une conduite d'eau forcée permet de
  faire tourner des turbines pour produire de l'électricité.
\item
  Un ascenseur dispose d'un contrepoids qui descend pour faciliter
  l'ascension de la cabine.
\item
  Le mouton Charpy (figure de droite) accélère lors de la descente
  (conversion de l'énergie potentielle de pesanteur en énergie
  cinétique).
\end{itemize}
\end{minipage} &
\includegraphics[width=2.34003in,height=2.9346in]{media/image128.jpeg} \\
\bottomrule
\end{longtable}

\hypertarget{stockage-inertiel}{%
\subsection{Stockage inertiel}\label{stockage-inertiel}}

\begin{longtable}[]{@{}
  >{\raggedright\arraybackslash}p{(\columnwidth - 2\tabcolsep) * \real{0.50}}
  >{\raggedright\arraybackslash}p{(\columnwidth - 2\tabcolsep) * \real{0.50}}@{}}
\toprule
\endhead
\includegraphics[width=2.18397in,height=1.45726in]{media/image129.jpeg}
& Il est possible de stocker de l'énergie sous forme d'énergie cinétique
en accélérant la vitesse de rotation d'un disque. Le freinage du disque
permet de la récupérer. Exemple~: Equipe certaines voitures de course
hybride pour récupérer l'énergie du freinage plus efficacement qu'une
batterie. Voiturettes à friction.

De manière plus instantanée les volant d'inertie sur les moteurs
thermique permettent d'assouplir le fonctionnement en lissant la vitesse
de rotation. Effectivement dans un moteur 4 temps l'énergie est produite
lors de la détente tandis que les 3 autres phases sont réceptrices
(aspiration compression échappement). \\
\bottomrule
\end{longtable}

\hypertarget{ressorts}{%
\subsection{Ressorts}\label{ressorts}}

\begin{longtable}[]{@{}ll@{}}
\toprule
\endhead
\includegraphics[width=1.08236in,height=0.91735in]{media/image130.jpeg}\includegraphics[width=1.01438in,height=1.01438in]{media/image131.png}
& Les ressorts permettent de stocker de petites quantitées d'énergie
mécanique, par exemple dans une montre mécanique. Il permettent aussi
d'obtenir une fourniture rapide d'énergie lors de leur détente par
exemple dans un système d'allumage d'arme a feu. \\
\bottomrule
\end{longtable}

\hypertarget{energie-thermique}{%
\subsection{Energie thermique}\label{energie-thermique}}

\begin{longtable}[]{@{}ll@{}}
\toprule
\includegraphics[width=0.53175in,height=1.21236in]{media/image132.jpeg}
& Pour simplifier la distribution, l'énergie thermique est en général
stockée dans des fluides, comme de l'eau dans les systèmes de chauffage
de maison. De l'huile est aussi fréquemment utilisée pour véhiculer les
calories dans un système (moteur thermique par exemple). Elle présente
l'avantage de ne pas être corrosive et de lubrifier automatique les
pompes, moteurs et vérins. \\
\midrule
\endhead
& \\
\bottomrule
\end{longtable}

\hypertarget{fonction-alimenter}{%
\section{Fonction Alimenter}\label{fonction-alimenter}}

L'alimentation est un composant ou un ensemble de composant nécessaire
lorsque la source d'énergie brute n'est pas conditionnée correctement
pour le convertisseur du système.

\hypertarget{alimentation-uxe9lectrique}{%
\subsection{Alimentation électrique}\label{alimentation-uxe9lectrique}}

\hypertarget{transformateur}{%
\subsubsection{Transformateur}\label{transformateur}}

\begin{longtable}[]{@{}ll@{}}
\toprule
\endhead
\includegraphics[width=1.43501in,height=1.47752in]{media/image133.jpeg}
\includegraphics[width=1.69658in,height=1.27314in]{media/image134.jpeg}
& Un transformateur permet de modifier les niveaux de tension entre 2
circuits électriques. En général l'électricité alternative est stockée à
des niveaux de tension élevé pour diminuer les pertes par effet Joule
dans les conducteurs. Par exemple 400000V pour les longues distances
jusqu'à 220V pour les circuits des particuliers. Souvent les machines
fonctionnent sur des tension plus faible, 24 ou 48V nécessitant de
nouveau une conversion par un transformateur supplémentaire. Les
transformateurs sont constitués de 2 bobinages différents et d'un cœur
ferromagnétique. \\
\bottomrule
\end{longtable}

\hypertarget{alimentation-uxe9lectrique-uxe0-duxe9coupage}{%
\subsubsection{Alimentation électrique à
découpage}\label{alimentation-uxe9lectrique-uxe0-duxe9coupage}}

\begin{longtable}[]{@{}ll@{}}
\toprule
\endhead
\includegraphics[width=1.30305in,height=0.87618in]{media/image135.jpeg}
& L'alimentation à découpage est une technique efficace, économique et
légère d'adapter une source d'énergie électrique à un niveau de tension
souhaité. Le signal source est haché par des transistors type MOSFET
pour obtenir le niveau de tension souhaité en sortie. Une régulation
permet d'adapter le niveau de découpage à la charge en sortie. Des
capacités permettent de lisser les signaux électriques. En général la
tension de sortie est plus faible que la tension d'entrée «~buck~» mais
il existe aussi des modèles qui remonte la tension «~boost~». \\
\bottomrule
\end{longtable}

\hypertarget{ruxe9gulateur-de-tension}{%
\subsubsection{Régulateur de tension}\label{ruxe9gulateur-de-tension}}

\begin{longtable}[]{@{}ll@{}}
\toprule
\endhead
\includegraphics[width=1.86538in,height=1.17044in]{media/image136.png} &
Le régulateur de tension est un composant électronique économique
permettant d'abaisseur une source de tension a une tension régulée plus
faible. En général les puissances sont faibles (alimentation d'un
microcontrôleur en 3,3V régulé sur une source USB 5V ou une batterie
LiPo 7,4V par exemple). \\
\bottomrule
\end{longtable}

\hypertarget{alimentation-pneumatique-et-hydraulique}{%
\subsection{Alimentation pneumatique et
hydraulique}\label{alimentation-pneumatique-et-hydraulique}}

\hypertarget{ruxe9gulateur-de-pression}{%
\subsubsection{Régulateur de pression}\label{ruxe9gulateur-de-pression}}

\begin{longtable}[]{@{}ll@{}}
\toprule
\endhead
\includegraphics[width=1.23846in,height=2.19758in]{media/image137.png}
\includegraphics[width=1.34257in,height=2.20565in]{media/image138.png} &
Le régulateur de pression permet d'obtenir une pression régulée à une
valeur souhaitée dans un circuit pneumatique à partir s'une source de
pression disposant d'un niveau de pression supérieur. En général
l'énergie pneumatique provient de pompes et de réservoir de pression
dont la pression peut varier. \\
\bottomrule
\end{longtable}

\hypertarget{systuxe8mes-de-conditionnement}{%
\subsubsection{Systèmes de
conditionnement}\label{systuxe8mes-de-conditionnement}}

Il est nécessaire de conditionner le fluide avant de la faire circuler
dans le circuit. Dans le cas de l'énergie pneumatique, il est
indispensable de s'assurer de la pureté de l'air ainsi que d'un faible
taux d'humidité. Pour cela on utilise d'une part des filtres permettant
de filtrer l'air entrant dans le réseau en amont et en aval du
compresseur. Il est aussi nécessaire d'utiliser d'un
refroidisseur-assécheur permettant de réduire le taux d'humidité. Dans
le cas d'un système hydraulique, le fluide est filtré afin d'éliminer
les impuretés.

\begin{longtable}[]{@{}llll@{}}
\toprule
\includegraphics[width=2.25581in,height=1.41748in]{media/image139.png} &
\includegraphics[width=0.92072in,height=0.55118in]{media/image140.png} &
\includegraphics[width=0.9327in,height=0.55118in]{media/image141.png} &
\includegraphics[width=1.06595in,height=0.62992in]{media/image142.png} \\
\midrule
\endhead
\emph{Systèmes de filtrage} & \emph{Symbole d'un filtre} & \emph{Symbole
d'un lubrificateur} & \emph{Symbole d'un déshydratateur} \\
\bottomrule
\end{longtable}

\hypertarget{systuxe8mes-de-suxe9curituxe9}{%
\subsubsection{Systèmes de
sécurité}\label{systuxe8mes-de-suxe9curituxe9}}

\begin{longtable}[]{@{}
  >{\raggedright\arraybackslash}p{(\columnwidth - 2\tabcolsep) * \real{0.50}}
  >{\raggedright\arraybackslash}p{(\columnwidth - 2\tabcolsep) * \real{0.50}}@{}}
\toprule
Afin de maîtriser la pression dans les conduites, on peut avoir recours
à des manomètres afin d'avoir une information sur la pression. Les
régulateurs de pression permettent quant à eux d'évacuer l'air du
système lorsque la pression est trop grande. Les limiteurs de débit
permettent de maitriser le débit de fluide.

Les systèmes de clapet anti-retour permettent d'imposer le sens de
circulation d'un fluide. &
\includegraphics[width=2.11323in,height=1.26713in]{media/image143.png} \\
\midrule
\endhead
& \emph{Régulateur de débit} \\
\bottomrule
\end{longtable}

\begin{longtable}[]{@{}ll@{}}
\toprule
\includegraphics[width=2.81395in,height=1.12709in]{media/image144.png} &
\includegraphics[width=2.79283in,height=1.15116in]{media/image145.png} \\
\midrule
\endhead
\emph{Schéma de compresseur intégré} & \emph{Unité
filtre-mano-régulateur-lubrificateur} \\
\bottomrule
\end{longtable}

\hypertarget{schuxe9ma-composants-pneumatiques-et-hydrauliques}{%
\section{Schéma -- Composants pneumatiques et
hydrauliques}\label{schuxe9ma-composants-pneumatiques-et-hydrauliques}}

\includegraphics[width=6.31034in,height=9.39481in]{media/image146.png}

\hypertarget{ressources}{%
\section{Ressources}\label{ressources}}

{[}1{]} http://www.festo.com.

{[}2{]} Caterpillar -- Pelles hydrauliques 374 D
Lhttp://s7d2.scene7.com/is/content/Caterpillar/C633539.

{[}3{]} http://www.defense.gouv.fr/.

{[}4{]} http://joho.p.free.fr/.

{[}5{]} http://www.espaceoutillage.com/.

{[}6{]} http://www.directindustry.fr/.

{[}7{]} Patrick Beynet, Fonctions du produit -- Technologie pneumatique
-- hydraulique pour les systèmes automatisés de production. Lycée
Rouvière Toulon.

{[}8{]} J. Perrin, F. Binet, J.-J. Dumery, C. Merlaud, J.-P. Trichard,
Automatique et Informatique Industrielle -- Bases théoriques,
méthodologiques et techniques, Éditions Nathan Technique, 2004.

{[}9{]} Guide des Automatismes Industriels.

{[}10{]} Cours «~Préactionneurs~». La Martinière Monplaisir.

Documentation additionnelle~: Stéphane Genouel stephane.genouel.free.fr
et s2i.chateaubriand.free.fr

\end{document}
